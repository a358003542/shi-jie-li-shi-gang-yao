% !Mode:: "TeX:UTF-8"

\documentclass[12pt,oneside]{book}

\usepackage{mybook} 
\usepackage{mybookcover}


\title{世界历史纲要}
\author{Wander}
\hypersetup{
    pdftitle={世界历史纲要},
    pdfauthor={Wander},
    pdfcreator={Wander},
    pdfsubject={历史},
}
  

\begin{document}
\makemytitle

\flypage{感谢上帝}


\frontmatter 
\addchtoc{序言}
\chapter*{序言}
历史作为人文学科,还真是一门作者主观性很强的领域,我无意于历史学研究,也无意于要发表出什么惊世骇俗的观点,本来就想随便找一本世界通史,对世界历史基本脉络有个了解,然后择个人感兴趣的重点内容,以为节点,挂在对应的位置上即可。

这对我投资认知世界还是很有必要的,个人也无意要说出一个观点,让大家都拍手叫好,对对对,是这样的,个人觉得对,就行了。但找来的历史书籍大多都不让我满意:

\begin{enumerate}
\item 我对太古早的那些历史琐事不感兴趣,但糟糕的是大多历史书籍都在讨论几百年前甚至几千年前的一些事情,真实性不论,就算是真的大多对现在也是影响微乎其微的了。
\item 一般的世界通史对最近的历史有所谈及,但讨论的内容太少了太浅了,根本不够用啊。
\item 别指望有一本客观的历史书籍,包括那种大纲性质的,历史作为人文学科,作者风格主观叙事倾向都太明显了,我想找一本够客观的,差不多够用就行,但想一想觉得还是算了吧,别幼稚了,这怎么可能。
\item 在叙事上我可能会更加关注经济上的一些事情,甚至会将其看作一个核心底层逻辑,几乎很难找到那种对我胃口的叙事风格的书籍。因此本书如果一定要分类,我会推荐分类到 \verb+经济 - 经济史+ 。
\end{enumerate}





\addchtoc{目录}
\setcounter{tocdepth}{2}    
\tableofcontents



\mainmatter
\part{西欧崛起的时期}
时间是1500年到1763年。

\part{西方占据优势的时期}
时间是1763年到1914年

\chapter{第一次工业革命}

\part{1914年至今}

\chapter{第一次世界大战}


\chapter{第二次世界大战}


\chapter{第二次工业革命}
第一次工业革命是节省劳力,第二次工业革命是节省脑力。



\appendix
\part{附录}
\chapter{术语释义}



\backmatter
\chapter*{参考资料}
\addchtoc{参考资料}
\begin{thebibliography}{99}
\bibitem[全球通史]{全球通史} 《全球通史:从史前到21世纪:第7版新校本》by [美]斯塔夫里阿诺斯 at 2020.
\bibitem[全球经济史]{全球经济史} 《全球经济史》 by [英]罗伯特·C·艾伦 at 2015.

\end{thebibliography}


\end{document}


