% !Mode:: "TeX:UTF-8"

\documentclass[12pt,oneside]{book}

\usepackage{mybook} 
\usepackage{mybookcover}


\title{世界历史纲要}
\author{Wander}
\hypersetup{
    pdftitle={世界历史纲要},
    pdfauthor={Wander},
    pdfcreator={Wander},
    pdfsubject={历史},
}
  

\begin{document}
\makemytitle

\flypage{感谢上帝}


\frontmatter 
\addchtoc{序言}
\chapter*{序言}
历史作为人文学科,还真是一门作者主观性很强的领域,我无意于历史学研究,也无意于要发表出什么惊世骇俗的观点,本来就想随便找一本世界通史,对世界历史基本脉络有个了解,然后择个人感兴趣的重点内容,以为节点,挂在对应的位置上即可。

这对我投资认知世界还是很有必要的,个人也无意要说出一个观点,让大家都拍手叫好,对对对,是这样的,个人觉得对,就行了。但找来的历史书籍大多都不让我满意:

\begin{enumerate}
\item 我对太古早的那些历史琐事不感兴趣,但糟糕的是大多历史书籍都在讨论几百年前甚至几千年前的一些事情,真实性不论,就算是真的大多对现在也是影响微乎其微的了。
\item 一般的世界通史对最近的历史有所谈及,但讨论的内容太少了太浅了,根本不够用啊。
\item 别指望有一本客观的历史书籍,包括那种大纲性质的,历史作为人文学科,作者风格主观叙事倾向都太明显了,我想找一本够客观的,差不多够用就行,但想一想觉得还是算了吧,别幼稚了,这怎么可能。
\item 在叙事上我可能会更加关注经济上的一些事情,甚至会将其看作一个核心底层逻辑,几乎很难找到那种对我胃口的叙事风格的书籍。因此本书如果一定要分类,我会推荐分类到 \verb+经济 - 经济史+ 。
\end{enumerate}





\addchtoc{目录}
\setcounter{tocdepth}{2}    
\tableofcontents



\mainmatter

\part{前言}
\chapter{历史的原则}
下面讨论的原则是普遍性的一般性的原则,偶尔出现的某个极其罕见的特例不影响这里的讨论。这些特例从整个历史大视角来看也是几乎可以忽略不计的。


\section{原则一:供求关系}
\uwave{供求关系决定了东西的贵贱价值。}

即使一个人不愿意为这个东西支付任何货币或者其他价值物,暴力获得该东西,该东西在此人心中的贵贱价值也不会受到半点影响,甚至没有货币和市场,也不影响这里的讨论。东西的贵贱价值仍由这个东西的潜在市场供求关系决定。这就是本原则称之为原则的原因,坚如磐石。


\section{原则二:趋利避害}
\uwave{趋利避害作为人的天性将主导一切。}

即使是一个疯子,再不怎么理性的人,蠢蛋,其思维行动的底层逻辑仍然是趋利避害,至少他自己是这样以为的,这就是本原则称之为原则的原因。

人们常常将这种现象更学术地称之为一个人说话做事是具有阶级局限性,这种阶级局限性是无法打破的,因为其底层逻辑就是原则二:人性的趋利避害。


\section{原则三:从众心理}
\uwave{一个观点说的人越多越看起来更正确。}

比如一个餐馆,生意兴隆去的人很多,那么人们会自然地认为这个餐馆的菜很好吃。从众心理也是广告总是保持其有效性的底层逻辑。人们常常将这种现象更学术地称之为一个人说话做事是具有时代局限性的。

对于原则三,人们会寻找各种特例,现象来说明原则三并不是那么坚不可摧。然而非常遗憾,历史现象看的越多,就会越来越发现,原则三坚如磐石。很多时候人们只是觉得似乎原则三的某一小部分被打破了,但即使是看起来被打破的那一小部分,也仅仅只是一种表象。处在整个历史洪流中的某个时代的某个人,身上打上了大量的烙印,这些烙印皆是因为从众心理。

对于从众心理推动的历史现象,应该从高于单独个体人的角度来看,即从整个社会的角度出发来看待这些现象。这也正是下面要讲的:社会驱动原则。



\section{原则四:社会驱动}
\uwave{历史现象是由社会整体驱动的。}

一般在描述历史现象的时候,人们都会夸大其词某些历史英雄人们,而如果这样表述,社会觉得应该是这样的,社会觉得这样会更好,人们会觉得实在是无稽之谈。对于整个社会的运作,不管是基于某种先验的作用,还是后验的实在的组织运作机制,将整个社会作为一个整体研究对象,总是比从个体人的角度出发要好一些,更贴近真实一些。这就好比人体内的细胞之于整个人体,从细胞的角度去理解整个人体运作之奥妙,是不可及的。


\section{最后的原则:丛林法则}
\uwave{丛林法则或者说:没有规则。}

动物的世界的法则就是弱肉强食丛林法则,人作为一种行为上更自由更脱离基因本能约束的动物,其构成的社会,假定人会比动物更加遵守某种规则,是没有道理的。更合理的推断是人类构成的社会比起某些动物同类构成的社会更加容易滑向彻底没有规则的丛林法则。之所以社会呈现出你看到的模样,是因为国家社会存在着各种奖惩机制,在这些奖惩机制照顾不到的地方,只有丛林法则。


\chapter{历史的分析}
\section{分析一:最低生活开支}
历史的分析需要跨越国家地区,跨越历史各个时期,很多情况都是不同的,可以利用最低生活开支 $ a $ 来获得某个通用性的数据。

基于最低生活开支,或者说贫困线的数据分析,对于某个工人的工资收入,将表述为最低生活开支的某个倍数。

\section{分析二:历史惯性}
一般来说关于一个地区跨越不同时代的数据,常常会发现某些东西很难改变,那么称这个东西具有某种历史惯性。

一个地区社会内部的分配制度常常具有很强的历史惯性。


\section{分析三:平均收入}
比如一个地区的平均收入是最低生活开支的 $ n $ 倍,这个数值和该地区的经济整体发展情况,社会分配情况密切相关,一般会具有很强的惯性。

这个 $ n $ 值有几个特定含义的标志值:

\begin{enumerate}
\item $ n $ 略微大于1,该地区大部分人陷入完全贫困陷阱中,地区经济状况具有较强的历史惯性,会一直被约束在略微大于1的位置,难以改变。
\item $ 1< n < 2 $ ,2这个数字意味着,该地区大部分人的收入勉强维持最低的生育延续线,该地区经济状况具有较强的历史惯性,任何进步因素都会被扼杀在摇篮里。
\item $ 1< n < 3 $ ,这个情况和上面的情况没有本质上的差异,属于该地区的好时候了,人口在增长,如果该地区文化有男尊女卑的倾向,那么这个时候女性会更加倾向于居家。该地区经济状况具有较强的历史惯性,但可能有一些进步因素在萌生。中国地区有名的历史周期律,其表现就是人们的平均收入在王朝好的时候会达到3点几的样子,然后就调转趋势向下陷入贫困中。
\item $ 2 <n < 4$ ,这个情况和上面的情况一样整个经济都属于农业型经济,不同的是该地区内部分配制度更公平政治更清明,会让这个数值更高一点,更加稳定在3到4之间波动。
\end{enumerate}






\part{殖民商业经济时期}

\chapter{时期概览}
时间是1492年到1785年。

这一时期以哥伦布航海探险大发现为起点。以1785年第一台蒸汽机驱动的纺织厂在英国诺丁汉建成,标志着纺织业从水力时代进入蒸汽时代,即第一次工业革命开始点为本时期的结束点。

这一时期经济:本国仍然是传统的封建农业经济,新增的经济风口是远洋航运商业贸易,并以此建立各个新的殖民地,根据当地的特色经济资源,将其纳入到全球商业经济网络中。

这一时期的政治:本国政治仍然是传统的封建君主制,其他政治事件主要围绕着海上霸权,即远洋航线控制和殖民地经济控制上展开。



\chapter{1492年哥伦布发现美洲大陆}
1492年,哥伦布发现美洲大陆。标志着本部分讲述的殖民商业经济时期的开始。

\chapter{1517年新教诞生}
1517年马丁·路德发表《关于赎罪券效能的辩论》标志着新教的诞生。

马丁·路德发动欧洲基督教改革运动,创立基督教新教。核心思想是因信称义,极大地削弱了原基督教教皇权威。教会势力渐渐退出历史主流舞台,欧洲的国家政权势力逐渐成为主流。基督教新教在各个国家有着不同的发展,比如英国的国教,则是试图在新教的基础上来完成政教合一。




\chapter{葡萄牙王位继承战争}
1583年西班牙合并葡萄牙标志着葡萄牙王位继承战争结束,从历史的角度来看,这标志着西班牙和葡萄牙的争霸以西班牙胜出结束。

但因为几年后英国就打败了西班牙,所以使得这一战争在历史上就显得不那么重要了,更不用提后面的葡萄牙王政复辟战争,其结果就是1640年葡萄牙独立了。



\chapter{英国击败西班牙舰队}
1588年,英国击败西班牙无敌舰队。当时西班牙合并了葡萄牙,因此等于英国将早期葡萄牙西班牙海上霸权拉下马。



\chapter{早期美洲东海岸移居活动}

\begin{bookref}[frametitle={\cite{美国四百年}}]
1616年,这群作为早期移居美洲的英国清教徒载入史册的人,原本是居住在荷兰的一个流亡的独立派团体。这些信徒中最早的成员于1608年逃离英国,最初落脚在阿姆斯特丹。几年后,他们辗转进入荷兰内地城市莱顿。...莱顿是一个“和善、漂亮的城市”,但是独立派信徒主要从事的是需要“艰苦而持续劳动”的工作,其中很多人从事布匹加工。

荷兰对他们的宗教迫害相对英国来说很少,不过在旅居荷兰十二年之后,一些人看到了宗教自由面临的潜在风险,于是这些分离主义者开始物色另外一个栖身之处,主要目的是改善经济境遇。

对于这些独立派信徒来说,最初促使他们离开英国前往荷兰的动机是过上好日子,并吸引更多基督教徒追随他们。然而,后来,莱顿的这些先驱者从事的苦力营生和“高强度劳作”彻底吓退了那些潜在的追随者。据布雷德福说,当时很明显的是,“一些人宁愿选择英国的监狱,也不愿意选择荷兰这些艰苦条件下的自由”。另外,他们中的年长者开始死亡,因为繁重的劳作往往会让这一宿命提前到来。同时,这些信众的儿女,“要分担父母亲的一部分重担”,不得不在类似其前辈要忍受的条件下劳动。如果这一切还不够糟糕的话,荷兰的“多重诱惑”还会吸引那些刚刚步入成年阶段的人远离教堂,走上“放纵”的道路,做出可能“亵渎神”的堕落之事。这些信众中的年长者清晰地意识到,如果没有会众的壮大,他们将在一代人的时间里被世俗的荷兰社会同化;这场实验最终将悄无声息地失败。总而言之,看起来,对他们宗教原教旨主义威胁最大的不是政治迫害,而是严峻的经济形势。

解决方案是再次迁移。最初的想法是改变方向,前往“地域广阔,人烟稀少的美洲地区”。这一提议引起了激烈的内部争论,主要是因为一些人想象的恶劣天气、野人、疾病、饥荒和“赤身裸体”的土著。此外,还有一个风险需要小心应对:在美洲西班牙领地附近落脚的想法被排除了,因为信仰天主教的“西班牙佬可能和美洲野人一样残忍”。其他美洲土地,大多数地方被英国声索\footnote{声明索取某地区领土主权。},剩下的相对来说面积小很多的地区被荷兰声索,所以,这就只剩下了两个选择。不管选择哪一个国家的领地,都要与对方谈判,获得许可。

当然,选择英国领地的可能性是一个戏剧性的嘲讽:当初逃离英国的信徒现在要考虑同先前迫害他们的那个国家友好谈判。不过,这个过程并非一帆风顺。

早在1606年,英国就向一个名为“伦敦弗吉尼亚公司”的私人风险项目颁发了特许证。虽然新世界的这一风险项目的内部运作由该公司内熟悉商业运作的人负责,但是监管和治理的权力仍然在英王手中。英王通过他设立的弗吉尼亚委员会来行使这些权力。和大多数新发起的风险项目一样——不管这些项目是海外项目还是国内项目——弗吉尼亚公司的开局很不妙。在其成立的头十年里,它数次亏损得一无所有,不得不再次筹资。更为糟糕的是,绝大多数被送往海外的劳工都悲惨地死去。

在最初的十年运作期里,弗吉尼亚公司饱受各种挫折和困难。当莱顿的独立派教徒派两人去伦敦,与他们商谈在其领地上落脚的事宜时,该公司喜出望外,就像是生意惨淡的店主看到了当天唯一的顾客。弗吉尼亚公司作为商业实体,能否盈利取决于殖民地的经营情况。那里急需大批孤注一掷,将生死置之度外的人。荷兰来的那两个教徒的迫切心情正好与弗吉尼亚公司的急迫心情不谋而合。不过,独立派信徒移居弗吉尼亚还有一个障碍:他们需要明确的许可,以便在那里从事宗教活动。看重经济效益的弗吉尼亚公司对眼前的机会非常乐观,向他们保证说,这个问题是小事一桩,只要英王例行公事的批准。然而,事实并非如此。这个过程不断拖延。弗吉尼亚委员会认为,如果他们批准独立派教徒在海外从事宗教活动,就会在某种程度上破坏英王陛下禁止其在英国从事宗教活动这一禁令的权威。后来,通过弗吉尼亚公司的斡旋,双方达成了妥协。英王的弗吉尼亚委员会既不批准也不禁止他们在海外从事宗教活动,前提是这些来自莱顿的信徒必须服从英王的管辖。

这种有意识的含糊成为英国政府代理人和那些信徒之间的中间地带。弗吉尼亚公司为莱顿的信徒授予了一个“许可证”,允许他们落脚新世界。实际上,不同于传统说法,最初移居美洲的清教徒根本不用躲避英王的迫害,他们是自愿向海外拓展英王主权的。

获得许可之后,身在荷兰的教派长老们现在面临着同样复杂的资金问题。虽然弗吉尼亚公司有能力提供许可,但没有能力提供远洋航行所需的资金,教友们需要自己筹资。对于谨慎保守的有钱人来说,船只、水手和给养所需的支出不菲,作为一项投资来说风险太大,尤其是作为海外风险项目,很可能血本无归。后来,解决方案似乎是物色愿意放手一搏的有钱人——想要获得巨额收益,不计较一两个项目失败。

正在这时候,一个荷兰商业机构向这些教徒抛出了橄榄枝,与弗吉尼亚公司的方案形成竞争关系。听说他们与弗吉尼亚公司的谈判后,这个荷兰机构劝说莱顿的独立派教徒前往荷兰在美洲的殖民地。然而,有了弗吉尼亚公司颁发的居留许可证,提供资金这件事就落到了英国的投资者身上。具体地说,一个叫托马斯·韦斯顿(Thomas Weston)的项目发起人,代表伦敦商业风险投资协会,辗转前往莱顿,百般游说独立派牧师约翰·罗宾逊(John Robinson)。韦斯顿劝说这位牧师,称自己“可以让朋友们投资”,于是双方草拟了正式的条款。用现在的话说,那些条款相当于一个意向书或投资条件说明书,一个需要根据最终谈判进行充实的框架。现代创业人士都知道,从协议双方讲明各自目标到最终达成协议,其间充满着越来越多的焦虑和对意志的考验,往往会持续到达成协议的最后一刻。1620年“五月花号”的筹资活动也不乏各种争执和波折。

...

风险投资在17世纪之前的很长时间里就已经成了气候。早在1505年,一个叫作“英格兰商业风险投资协会”的组织就获得了官方的批准。这种协会不是将资本或资源汇集在一起的正式组织,而是松散的行业团体,个体成员可以有选择地参与团体的风险项目。后来(同一个世纪),海外风险项目所需的资金越来越多,恰逢股份公司出现并推动了这些项目的发展——相对于出资人相互熟稔、相对封闭的合伙关系,“股份制”指的是股权可向任何人转让的股东关系。

除了股权可转让,商业法规的不断进步也为减少个体风险投资者的个人责任提供了有利条件——如果项目亏损,投资者可以不用承担超过其初始投资的亏损金额。有限责任概念是法律领域的一个新事物,在自由市场中并不自发或有机地存在。有限责任可以让投资者获得无限的利润潜力,而限制可能的亏损程度,这大大增加了航海探险投机活动的吸引力。有限责任形式不一定和海外殖民有关。主权国家经常向国民授予独占捕鱼权、勘探权,向私营实体授予贸易路线。政府通过向私营实体颁发特许证,鼓励私人资本投向海外风险项目,从而为国内创造经济效益。

有限责任对于鼓励投资,实现上述目标发挥了关键作用。对于远在遥远海域的船只和持续时间长达数月或数年的海外贸易项目,由于其商业性质,身在英国的投资者很少有参与决策的机会。这就强化了对公司制的迫切需求,因为在公司制下,被动型投资者无须对未知债务承担责任。同时,这种商业项目所涉及的航行距离和海外使命持续时间,决定了这些项目需要数量庞大的资本,远远超过了任何单个投资者的风险承受能力,不管他多么富有。股份公司可以让很多投资者出资参股某个项目,获得相关利润。促使英国企业采用股份制的最后一个推动因素是1553年成立的一家叫作“俄罗斯公司”的股份公司。当时,众多风险投资者以每股25英镑的价格,总共投入6000英镑;这标志着法人形式首次被用于海外投资。

从那时起,连英国的私掠者(经政府批准可以抢夺敌国海上货物的海盗船)都开始采用股份公司形式向风险投资人筹集资金。私掠者想分散风险还有另外一个原因。对于从事这一行当的个人来说,如果私掠活动的出资群体中有足够多的声望很高的人物,那么因政治风向发生转变(即使当初被政府批准)而被控犯罪的风险会大大降低。这些私掠者根本不是传说中戴着黑眼罩、肩膀上站着鹦鹉的草莽人物。有关每次私掠活动的会计报表都会详细记录动用船只的吨位、投入的资金、参与的人员和船只数量。在有关弗朗西斯·德雷克爵士(Sir Francis Drake)组织的一次私掠活动的财务报表中,赫然记录着这次冒险总投资额为5.7万英镑,动用了21艘船和1932名水手。W. R. 斯科特(W. R. Scott)深入研究那个时代的众多股份公司之后,认为这种公司结构的灵活性为经营活动提供了两种优势,即投资的多样化和风险分摊,尤其是在私掠方面。私掠活动对巨额亏损的容忍程度催生了现代风险资本的基本理念。

假如,一个出资人打算在私掠项目上投资2000英镑,那么只够为一艘排水量为200吨或两艘更小的船只配备相关设备。这样,如果仅凭一己之力,私掠活动的力量就会过于薄弱,难以获得有价值的斩获。但如果他用这些资金和很多人共同投资多次大规模的私掠行动,即使某次活动完全失手,他仍旧可能从其他活动的利润中获得不菲的收益。

收益不菲,确实如此。德雷克的这次私掠活动获得了4700\%的回报,也就是其投入资本的47倍。在权衡投资机会的风险投资者看来,这一奇高的收益率是极为诱人的,要知道,他们可不是保守的伦敦银行家。

17世纪初,经过一场极为残酷的战争,英国经济遭到重创,陷入萧条。除了英国本土之外,其他地区经济形势都很好。1600年,东印度公司成立。数年后,伦敦的弗吉尼亚公司成立。公司成立没多久,就印制了吸引风险投资者的小册子。弗吉尼亚公司的宣传册刻意淡化了所有不利因素。在有关印第安人可能带来的危险方面,宣传是这样说的:“他们大都善良和气,无微不至地关照我们。”广告中还详细叙述了可以运回英国做木材的各种树木,蕴藏丰富宝藏的“从未探察过的”山脉和辽阔“富饶”的土地。接下来,小册子号召广大英国同胞发扬爱国精神,积极参与,说他们必须“前往世界的每个角落,解决英国的物资匮乏问题,用一个王国的物资供应另一个王国”。然后,小册子讲到国内严重的失业状况,到处是“大批闲人”,可以将他们送往海外从事这一项目。

接着,小册子开始说眼前的项目。对于想留在英国的风险投资人,弗吉尼亚公司一股的价格是12英镑10先令。对于移居者,也就是那些真正愿意去弗吉尼亚的人,不用出任何资金就可以得到公司的一股,前提是在弗吉尼亚殖民地做满七年工。另外,公司负责提供伙食费、材料费和殖民地的维护支出。公司拥有所有生产资料,对所有经营活动享有垄断权。在七年结束之际,包括已开垦土地在内的公司资产将被分配给所有股东。对于很多经济拮据,为生活苦苦挣扎的年轻人来说,只要出力气就可以在新世界获得一块土地,对他们具有相当的诱惑力,尤其是当他们看到很多富人也愿意以超过12英镑的价格购买这些股票时。

轻松赚大钱的梦想往往很容易破灭。弗吉尼亚公司后来变成了一场灾难。头几批到达的劳工饱受疾病、饥荒、严寒和印第安人袭扰的折磨。随后几批给养船到达的劳工被眼前长期食不果腹的同胞吓坏了,早期到来的一批人干脆逃到了印第安人中间。连续经历了几次失败后,老股东不愿意继续投钱,公司不得不物色新股东,并进行多次重组。1614年,公司财务状况极度恶化,急需大批资金注入。再加上,他们建立的殖民地的状况也一团糟,于是,人们群情激奋,要求撤销该公司的特许权,上交王室。弗吉尼亚公司的律师理查德·马丁(Richard Martin)向下议院申请救助,希望英国财政部能提供应急资金援助。

因此,当那两个来自荷兰的访客向弗吉尼亚公司打听如何落脚新世界时,后者愿意招收任何甘冒生命危险,且能忍受恶劣工作条件的人。

...

获得弗吉尼亚公司和英王的许可近两年半后,莱顿的独立派教徒为远航做了最后的准备工作。威廉·布鲁斯特(William Brewster)和约翰·卡弗(John Carver)代表所有信众,动身前往伦敦签订最终条款。

股票定价为每股10英镑。条款类似于弗吉尼亚公司的条款。愿意去新世界耕作的人可以获得一股,愿意投资10英镑的任何人也可获得一股。风险项目拥有殖民地的所有资产和经济权利。七年结束之际,总资产按照所有权比例在所有股东之间进行分配。

然而,问题出在细节中。就在距离原定启程日仅剩几个星期之际,商业风险投资协会更改了协议条款。在最初的条款里,这些信徒每周为该项目工作四天,剩下两天时间用以“个人工作”,一天是安息日休息。后来,投资者坚持要求人们一周给这个项目工作六天。更令人气愤的是,那些劳工给家人建的房子也不属于他们自己,而是属于这一风险项目。

这些信徒的精神领袖约翰·罗宾逊非常不满。他写信给他的代表约翰·卡弗,抱怨久久无法租到船。罗宾逊发现,七年到期后将房屋作为项目资产来分割的条款尤其显得吝啬,因为,该项目的主要利润来源是“捕鱼、贸易等工作”。接着,他请原定牵头进行这次远航的卡弗认真考虑劳作七年且“没有一天不劳动”是一种什么感觉。然而,已经太晚了。代理人已经代表所有信徒在协议上签了字。

在筹集到1200英镑之后,相关各方立刻开始腌制牛肉,储备啤酒、饮用水和海上航行需要的其他物资。在荷兰,随着该项目的一些股份在独立派信徒中发售,他们用这些钱购买了排水量为60吨的小型船只“佳速号”,打算抵达新世界后将这只船用于海岸贸易和捕鱼。在英国,人们租下了排水量为180吨的“五月花号”。

1620年7月的一天,“佳速号”矗立在荷兰港口城市德夫哈芬(现今的鹿特丹),准备载着部分独立派信徒前往英国,与停泊在那里的“五月花号”会合。其中的一个乘客,也就是未来的朝圣者领袖威廉·布雷德福,记述了那个极为伤感的一天:面对从附近的阿姆斯特丹赶来送别的亲友,那天“不知流下了多少泪水”。“德高望重的牧师满脸是泪,跪倒在地”,希望作为先驱者第一批出发的教友一切顺利。不久,“佳速号”徐徐离开了码头。就这样,“他们离开了生活将近十二年的那个优美宜人的城市,不过,他们认为自己是朝圣者”。

没过两天,在一路“顺风”的帮助下,这些朝圣者到达英国南安普敦港。看到“五月花号”的初始兴奋退去之后,手里的协议让人们很快郁闷起来。从荷兰来的信徒立刻对代理人认可的条款提出异议,尤其是工作日由四天变成了六天,以及自建房屋由个人财产变成了共有财产。商业风险投资协会的代表韦斯顿从伦敦赶来,以“确认协议条款”。但是,这些朝圣者不承认新增条款的约束力,虽然他们的代理人签了字。僵局持续了好几天。这时候,事先约定的最后一笔100英镑的资金到了出资日期。这笔钱本应由韦斯顿他们支付,用于让“五月花号”起锚离港,但韦斯顿拒绝出钱。信徒们只好卖掉了3000磅黄油,低价处理掉一些其他物品,迅速筹集了60英镑。

另外,朝圣者的虔诚信仰促使他们一致向其他股东表达了他们不让步的原因,其中包括和商业风险投资协会一起投入资金但仍然留在莱顿的很多信徒。他们提出,在新世界拥有自己的房子是促使他们前往那里的“重要因素”。不过,他们又安慰投资人说:“如果七年内不能产生大笔利润,我们大家还会继续合作。”虽然在其他方面对投资人百般安抚,但是这些朝圣者在房子所有权上丝毫不让步。韦斯顿也不让步。

卖掉一些给养获得资金后,1620年8月5日,“五月花号”和“佳速号”离开码头,驶向新世界。因为筹集资金和租船耽误了一些时间,所以启程时间比原计划稍晚。按照这个时间,他们可能在初秋抵达新世界,勉强有足够的时间为冬天做准备。屋漏偏逢连夜雨。“佳速号”名不副实——因为漏水不得不停靠最近的港口修理。漏水问题解决后,又发生了其他导致返航的问题;这一次,“佳速号”被彻底放弃。最初的一些乘客决定放弃这次前景不妙的旅程,剩余的乘客则转而乘坐“五月花号”。

9月5日,“五月花号”载着102名乘客,开始了驶向新世界的旅程。很多记载说这些朝圣者是因为宗教信仰原因前往新世界的;需要指出的是,“五月花号”上足有一半的乘客并非来自莱顿的独立派信徒,他们只不过是被投资者分配到那艘船上的移居者。

原本生活在相对自由的荷兰,现今在英国人的资金支持下,乘坐英国船员驾驶的从英国租来的船只,在英国国旗下驶往新世界——这一切都不像是逃离英国迫害的样子。政治难民一般不会在逃难前的最后日子里商谈未来七年的经济报酬和资产分配。所以,这些朝圣者从开始就根本不是难民,他们是一场投机活动的关键参与者,同时也是向新世界扩展英王主权的关键参与者。宗教自由只是整个计划中的一部分。

然而,这一风险项目,和弗吉尼亚项目一样,开局非常不顺。11月末,“五月花号”才抵达美洲海岸。除了抵达的季节过晚之外,他们还错过了目的地。弗吉尼亚公司当初授予的“特许证”规定的登陆地点在哈得孙河河口附近。然而,“五月花号”停靠地点却在河口以北220英里处的半岛旁。在派出一个探险队上岸查看海岸线,并物色建立定居点的位置后,大多数朝圣者仍然待在“五月花号”船上,等待探险人员回来。在“五月花号”出航期间,英国的官员们将弗吉尼亚北部的一块土地划给了新英格兰委员会。在不知情的情况下,这些朝圣移民正在为他们在新英格兰的第一个冬天做准备。

威廉·布雷德福在他撰写的历史书《普利茅斯开拓史》中将这一部分的标题定为“饥饿年代”。航行途中只死了一个船员,横渡大西洋结束之前,没有乘客死亡。但在人们等待探险队归来的过程中,死神光顾了那艘船。布雷德福回到船上后,发现妻子死了。而这只是等待着他们的严冬的前奏。2月底,这块新建殖民地的死亡人数达到了“一天两到三人”,而且已经持续了一段时间。布雷德福极为痛苦地说,“只有六七个人”身体无大碍,能够照料别人。到了3月,“五月花号”的乘客将近死了一半。英国谈判导致的耽搁,再加上“佳速号”漏水返航,让登陆新世界的时间推迟了整整一个月。而人们本可以用这些时间来准备应对未来几个月的寒冷天气。

在科德角挨过冬天之后,租来的“五月花号”在4月开始返航。看重经济效益的风险投资人原本期望这艘船返航时从新世界带回木材、毛皮和其他商品。但是,因为那些移居者在第一个冬天里的恶劣境遇,“五月花号”返回英国时基本上是一条空船。

很多投资者——尤其是托马斯·韦斯顿——很不满意,认为死亡率不是借口。在写给约翰·卡弗(当时的卡弗已被那些朝圣者选为总督)的信中,韦斯顿不无挖苦和不满地写道:“船上没有送回来什么东西,你们可真是不容易,人们对你们不满意也不冤枉你们。我知道你们的问题所在,与其说是手脚有问题,不如说是头脑有问题。”韦斯顿这封写给卡弗的信交由一艘叫“财富号”的新船送达,其目的地是普利茅斯。韦斯顿担心卡弗不能充分理解他的意图,还特意在信中告诫对方说,如果卡弗不能让这艘船满载而归,他就要切断他们的资金供应。然而,在“五月花号”回到英国和韦斯顿通过“财富号”寄送的信送达的这段时间里,卡弗去世了。

经历了冬季的痛苦之后,“财富号”的到来让人们大感欣慰。“财富号”除了带来急需的给养之外,还带来了35个移居这里的劳工。韦斯顿那封措辞很不客气的信被交到了新任总督威廉·布雷德福手中。不过,到了这个时候,形势已经开始好转。从出现第一丝春意开始,这些移居者花了好几个月来建造他们的居住区。更重要的是,他们第一次与印第安人有了直接接触。那个印第安人自称萨莫塞特(Samoset)。不可思议的是,与科德角附近从事季节性捕鱼的讲英语的渔民熟悉一些之后,萨莫塞特居然能够结结巴巴地说一些英语。他甚至在首次打招呼时主动走向前说“欢迎你们,说英语的人”。这句欢迎语让对方的紧张情绪大为缓解。

在这次接触的几天后,当地部落首领马萨索伊特(Massasoit)前来造访。这次会面意味着之后延续了二十年的个人关系的开始。印第安部落对普利茅斯殖民地的救助至关重要。具体而言,印第安人能够弄到一种可以在欧洲卖到很高价格的内陆商品:河狸皮。欧洲各国目标市场的有钱人对河狸皮制品趋之若鹜,导致河狸皮迅速成为新英格兰和旧世界之间的商业纽带。同时,跨大西洋贸易塑造了美洲土著和早期殖民者之间共生的经济纽带。因为进入内地的殖民者装备很差,所以印第安人没有领地会被侵占的忧惧,而是将英国人的殖民地看作贸易站。多年来,河狸皮是印第安人冬季御寒衣服的一部分,他们猎捕河狸的技术非常娴熟,因此在弄到殖民者渴望的这种高价值商品方面,印第安人具有很强的竞争优势。长途跋涉到人迹罕至的水塘捕捉河狸,不是一件容易的事情,殖民者们将这件辛苦活儿交给这些行家去做。

北美土著将另一部分价值添加到供应链中来:河狸皮需要大量的加工工作。肢解和扒皮落在了本地女人们的身上。将皮毛上的肉和肥油去掉之后,接下来是一个更消耗体力的任务:软化生皮,将粗糙的硬毛去掉,最终加工成适合欧洲上层人士穿用的光滑柔软的毛皮制品。有时候,这甚至需要穿着生皮一年,用汗水做软化剂,达到软化效果。

这一情况促成了普利茅斯早期殖民者的贸易机会。早在“财富号”抵达之前,殖民者就已经开始用一些简单的物品,比如毯子、玻璃珠、刀具和其他器皿,换取印第安人的河狸皮。印第安人没有铸造锃亮刀具或精致金属用具的技术,却能够用他们看来简单的猎捕和加工河狸来换取这些奢侈品。...

这些殖民者没有把韦斯顿的挖苦放在心上,将大量木材(以板材的形式)装上了“财富号”;更重要的是,他们还给船上装了“两大木桶河狸皮和水獭皮”。这批货物价值“将近500英镑”。这是一个利润颇高的套利交易:据朝圣者中的那位领导说,这些即将出口的东西是殖民者用“一些微不足道之物”换来的。那次航行的初始投资,大约是1200~1600英镑,毛皮和木材的价值提供了投资者第一年红利的30\%——这是为一群本质上是宗教原教旨主义者的人提供资金的坚实回报。不过,上天却不保佑“财富号”。那艘船在大西洋海面上航行时,遭到一艘法国私掠船拦截,货物被抢走。结果,和“五月花号”一样,“财富号”也空手回到了商业风险投资协会那里。

这一灾难性事件与出资人意外遇到的其他几个问题同时发生,并引发了之后的那些问题。如果那批货物安全抵达的话,销售利润可以用来投资殖民地来年所需的给养。现在,采购相关工具装备和另派补给船的成本,需要再进行一番筹资。而接下来的筹资活动难度很大,因为这一项目到目前几乎颗粒无收。同时,韦斯顿和另外一个出资人发生了争吵。一时间指控四起,诉讼不断。这场风波大大影响了韦斯顿在出资人中的威信,尤其是在需要进行额外一轮融资的情况下。不久,韦斯顿将自己的股份让渡给其他所有出资人,这场争执才算结束。

恰似海外项目出资人的纷争,北美殖民者也遇到了棘手的问题。按照最初的协议,定居点和公司的所有人都要参加种植劳动。给养船的到来一再推迟,即使到来,又常常带来比食物更多的需要喂养的人口。在这个时候,种植庄稼关系到人们的生存问题。然而,“人人为我,我为人人”的思想造成的粮食短缺,说明这种集体劳动方式是靠不住的\footnote{历史早就证明了在粮食生产上大锅饭注定是失败的。}。经过激烈争论,大家决定所有家庭都分得一份土地。每个家庭可以自由支配自己的劳动成果。殖民地总督说:“这个办法很成功,它让所有人都勤快起来。女人们积极主动地去地里干活,还带上小孩子帮她们往地里撒玉米种子,不再说孩子没力气不会干活的话了。”这种实验性的共产主义生产形式,至少在种庄稼方面,在这些朝圣者中结束了。

同时,国内投资人逐渐意识到,在项目期限内获得丰厚利润的希望很小。更糟糕的是,英国恶化的经济和政治形势让很多投资者资金紧张。“五月花号”出航不到三年,金融机构就开始瓦解。富有同情心的投资者詹姆斯·谢林(James Sherley)在给布雷德福的信中沮丧地说:“之前,不管哪次出海和交易,你和我们都是参与者和合作伙伴,那种时候再也不会有了。”另外,他更关心的是确保“我们的钱不要亏掉”。谢林估计,出资者对“不少于1400英镑”的公司资产有优先要求权(first claims,在现代风险资产术语中叫“优先清算权”)。又一段谈判开始了。

就在英国投资者内部,以及他们与朝圣者进行激烈谈判的这段时间,该殖民地仍然需要外部资金注入以维持生存。为了弥补资金缺口,这些朝圣者向贸易合作伙伴和经纪人张口借钱,利率超过了50\%。

众多利益群体经过将近两年的谈判,逐渐达成了一个解决方案。普利茅斯的殖民者不愿根据协议将房屋和田地作为公共财产进行分割;同时,身在伦敦的风险投资人也无意直接接手位于数千英里之外偏僻村庄的房产。经过复杂的讨价还价,双方达成一份协议。普利茅斯殖民地负债1800英镑,买入该风险项目的所有股权——这样,出资人就不再是该项目的股东。这使得殖民者可以自由地在内部分配房屋和田产。同时,一些朝圣者,以威廉·布雷德福牵头,承诺承担全部集体债务;从1628年开始,每年支付200英镑,九年共计支付1800英镑。为了缓解普利茅斯公民的偿债压力,布雷德福和他的团队申请到了该殖民地从事毛皮生意的垄断权。

实际上,这一垄断权含金量极高,但前提是没有竞争。到1628年,一船又一船的移居者抵达新英格兰。有的移居者,比如前来定居的清教徒,有宗教倾向,而其他人则是流动商贩。殖民地位于哈得孙河的荷兰人在北至康涅狄格河的地方建立了贸易站。法国人也不甘落后。对于以上各路人来说,毛皮都是至关重要的商品。北美土著居民继续他们作为猎人和毛皮加工者的角色,是跨大西洋贸易的关键人物。对于当地河狸来说,这可不是什么好消息。繁殖率低,再加上迁徙距离有限,这场毛皮争夺战几乎让新英格兰的河狸迅速灭绝。随着河狸的消失,当地印第安人基本丧失了价值,甚至成了危险因素。

市场形势的压力迫使布雷德福和信徒们重新与出资人谈判,使他们将皮毛生意深入进行下去以获得出路。1645年,当所有矛盾尘埃落定时,仍旧持股的出资人已屈指可数。“五月花号”的资金筹集和每次耗时数周的跨洋远航,形成了一个持续二十五年的风险项目。当那份协议有效期满时,当初船上的朝圣者们绝大多数已不在人世。

\end{bookref}

实际上五月花号项目不是弗吉尼亚公司的第一个风险项目,美国现在的弗吉尼亚州的名字不确定是否来自弗吉尼亚公司,但确实弗吉尼亚公司的第一个移民项目就落在了弗吉尼亚州的东南部,1607年,105名英国人来到了美国的弗吉尼亚州,建立詹姆斯敦殖民地。

而詹姆斯敦殖民项目也不是弗吉尼亚公司的第一个风险项目,此前的18个都定居失败了。至于詹姆斯敦殖民项目很快也失败了。1622年印第安人对詹姆斯敦定居点进行了大屠杀,300多名英国移居者丧生。

五月花号作为弗吉尼亚公司的风险项目,是偏离了原定计划的落脚点的,落在了现在美国的马萨诸塞州。那么五月花号是不是弗吉尼亚公司的最后一个风险项目呢,也可能不是,不过也不远了。上面提到1622年詹姆斯敦定居点惨剧就是一个例子,鉴于弗吉尼亚风险投资项目超高的死亡率和黯淡的前景,1624年詹姆斯一世收回了弗吉尼亚公司的许可权,之后就没有弗吉尼亚风险投资项目了,而是作为一个殖民地被英国管理着。

显然移民美洲的其他项目仍然在前仆后继的继续着,所以五月花号项目肯定也不是最后一个移民项目,那么美国的历史编写者们为什么要着重强调五月花号呢?也许仅仅是因为这艘船的出发和目的带有一点圣经上的浪漫气息吧。

五月花号上的移居者之后的命运怎样了似乎历史已经选择性遗忘了,因为真实的现在的美国历史的开端,是以东海岸陆陆续续一系列的移民定居点开始的,有的成功了,有的失败了,而五月花号,只是似乎多了一点点浪漫气质,多得不多,甚至有可能是失败的那一批,即使五月花号定居项目成功了,也只是这整个移民定居项目中不起眼的那个罢了。读者在推演美国历史的时候,最好以这样的视角来看待五月花号,当个典型略微了解下即可,别太看重,如果认为美国历史是以此定居点推演一生二慢慢发展起来的,那就错的离谱了。




\chapter{牛顿发表自然哲学的数学原理}
1687年,牛顿发表《自然哲学的数学原理》。

\begin{bookref}[frametitle={\cite{牛顿传}}]
牛顿是一个清教徒,这一点在他的整个学术生涯中影响着他。当时英国的大学是奉英国国教为圭臬的,即使王室复辟后,昭示容忍宗教的多元性——英国社会的主流思想一贯主张如此,但大学生还是被迫至少要在形式上服膺于英国国教,所以牛顿不得不把他的清教徒信仰藏在心中。

牛顿的宗教倾向无形中带给他不知多少的潜在问题,这让他与同学(那些正统派)之间树下多重藩篱,但清教徒的道德激励了他学习的决心,令他心无旁骛地全力学习。幼年时代被母亲遗弃,使他深受伤害以致情感无能,清教徒主义的节欲世界赐予他一个极好的借口,作为他对自己无心恋爱的解释。因为在清教徒的世界里,只有上帝和知识这两根精神支柱,而追求知识又是上帝赋予他的神圣使命。这两根支柱可以取代其他一切需要,他以清教徒主义和渴求知识的天性为引导,至少可以躲避性欲的需要。没有结婚或成家的心理压力,更抑制了他对物质的欲望。

...


牛顿的《原理》不仅整合了伽利略和开普勒的理论,成为单一的、内聚的,用数学表达、用实验支持的整套理论,同时也打开了工业革命的大门。牛顿除了解决潮汐如何产生、彗星如何划越天空等种种困惑人类的古老问题,还引入了较新的观念,譬如解释地球自转时的“摇晃”或岁差是因为地球上各点有不同的重力强度。《原理》为力学和动力学的研究奠定了基础,在其后的一个世纪内,引发了持久且真实的人类文明演变。若没有这样的了解,大自然的各种力就不会获得利用,而这正是工业革命的成就:它将人类从黑暗中,从对大自然的奇想中,拉往科技时代,拉向操控宇宙之力的时代。

\end{bookref}





\chapter{牛顿的造币厂岁月}
过去的历史记载和牛顿传记对牛顿的那段造币厂岁月是不愿提及或语焉不详的,在历史上更多的被人们谈起的是物理学家牛顿和他的巨著《原理》,而现代社会已经越来越重视从经济金融的角度来看待历史变迁,于是人们也开始越来越重视和讨论牛顿作为造币厂督办或厂长的那段岁月对历史的影响。

引用自瓦伦丁·博斯的《牛顿与俄罗斯》:
\begin{quotation}
1698 年前往英格兰游历的俄国沙皇彼得渴望见识该国的几样非凡之物:造船术、格林尼治天文台、铸币厂和艾萨克·牛顿。
\end{quotation}

实际上在十七世纪晚期,牛顿在学术圈的盛名也仅仅局限在学术圈,让牛顿在英国政坛崛起,获得自己梦寐以求的权力——当选皇家学会主席,被册封爵士,变得小小的富有,都是因为他在造币厂的出色工作。

\begin{bookref}[frametitle={\cite{牛顿新传}}]
1693年约50岁的牛顿有过一段时间的精神崩溃期,后来牛顿的思想恢复了平静、生活恢复了正常之后,又作了最后一次努力,想矫正他在《原理》中论述月球理论时遇到的一些顽固问题。可以说,这将是牛顿从事的最后一项主要的持续性科学活动。从1694年夏开始,他再次尝试解决月球理论问题。...但到头来,牛顿还是没能解决棘手的“三体”问题。而不解决这个问题,他就无法在月球理论上取得进展。
\end{bookref}


认为牛顿因为物理学上棘手的三体问题得不到突破,而突然转型变得对政治和世俗事务感兴趣了这是不对的。实际上牛顿一直就对政治权力很感兴趣,牛顿从天性上就是一个官僚,\cite{牛顿传}说的好,对于牛顿而言,他在科学上的动力就是求知欲,而知识至于他而言就是权力。

当然就算牛顿政绩不错,也必须站对位置,总的来说牛顿的政治觉悟还是不错的。这里不讨论牛顿真的政治主张,牛顿贪图的是政治权力,政治主张对他不是最重要的,他并不是一个政治主义者,所以后面我们会看到牛顿发现风头不对之后,很自觉的就退出政治舞台了。

当时的英国政治局面是这样的,1688年末詹姆斯二世逃离英格兰,威廉三世来到英国,新的政权新的班子,牛顿作为剑桥大学当选为了国会议员,此时的牛顿对于自己的当选都是有点吃惊的。而后他的投票确认詹姆士二世已经放弃了王位,于此时起牛顿也算是走进英国的政治圈子了,不过也就止步于此了,并没有收到额外的重用。牛顿在一篇法律废除权的分析陈述中认为国王的权力在人民之下,只有人民才有权力废除法律,这表明牛顿是持有辉格党人的政治观点的,其后牛顿便走进了辉格党的圈子。

1702年威廉三世去世,继承王位的安妮女王不喜欢辉格党,托利党受到重用,牛顿敏锐地察觉到了这种变化,其后在政治上保持低调。这种低调自保的态度是明智的,1703年当选皇家学会主席,1705年受封爵士,直到1724年因身体状况不佳主动放弃造币厂总监和皇家学会主席职位,这些政治权力的获得和保持除了基于他造币厂上出色的工作政绩,就是得益于明智的政治站队和之后明智的低调态度。【牛顿获得政治权力之后干的那些破事这里就不细说了】

查尔斯·蒙纳古,当时的财政大臣,同时还是皇家学会的主席,还是牛顿的学生,是不是牛顿的朋友不知道,但显然蒙纳古对于牛顿的学术成就是知晓的。他是这样说的:

\begin{quotation}
我非常高兴,因为我终于能向您证明我的友谊以及国王对您的功绩的赏识。国王已应允我任命牛顿先生为造币厂监督。这个职位最合适不过了,年俸约为五六百英镑,而事情不多,花销不大......
\end{quotation}

虽然当时英国财政上一堆乱事,但说蒙纳古是因为觉得牛顿能力出众,渴望他出山力挽狂澜,那是扯淡。更多的一个闲职“事情不多”,薪资颇丰“一年五六百英镑”。

当时的英国从货币市场上来讲有以下几个问题:

\begin{itemize}
\item 银币被剪钱问题:银币被人偷偷的减去边边角角来获利。提出的解决方案就是回收旧币铸造新币,在牛顿进入造币厂之前,铸造新币的工作已经着手推行了,但因为各种困难而进展缓慢,具体细节后面细讲。

\item 伪造货币问题:新币的铣边可以防止剪钱,不过不法分子仍然可以将银币完全熔化,再掺杂廉价材料重新铸造伪币来获利。一开始牛顿对打击伪造货币犯罪问题不太热心,觉得这不应该归造币厂管,推托无果之后,牛顿花了大力气做起了侦探来打击伪造货币犯罪。

\item 银子外流问题:新币走的是等值银子路线,银含量比较高,欧洲大陆政府权力较大,铸币话语权更大,所以铸币的时候一般会放点水。各个国家货币银含量不同,人们跨国贸易的时候必然采用的是等值货物交易逻辑,这个时候用金子来作为一般等价物是最合适不过的,还有从货物贸易的角度出发,因为各个国家货币银含量情况很不一样,人们在乎的也仅仅是该货币的银含量。我们现在假设有1克金子,在英国这边有旧币,银含量比较低,假设1g金子18s,这里单位是假定的,而新币假设1g金子21s;在法国这边是1g金子17s,还有一种新注水的货币1g金子15s。假设你是一个手头上有1g金子的商人,那么这个时候你首先想到的是跑到英国用1g金子收购新币,把新币熔成银子再跑到法国,那么你就可以兑换更多的金子。上面的讨论如果非得代入现在的货币面值,你可以认为1g金子在英国就是1英元,1g金子在法国就是1法元,也就是从等价物的角度来说,1g金子等于1英元等于1法元。继而有1英元等于21s银子或者18s银子,1法元有17s银子或者15s银子。读者这个时候肯定已经发现问题所在了,长此以往英国含银量较高的新币就好像走入一个黑洞一样必然不断地被民间熔掉。
\end{itemize}





\section{回收旧币铸造新币}

\begin{bookref}[frametitle={\cite{牛顿传}}]
我们可以想象,牛顿的到来给造币厂的领导带来了多大的震撼,因为他是所有人当中最勤快的一个。...在头几个星期中,每天凌晨4点,当压制机开始启动时,牛顿就在厂房里了,夜班开工前他又赶回来监督。有一阵子,他甚至住在工厂旁边为他准备的带有一个小院子的宿舍里。在历任的厂长中,没有一人做过这些事。

这间宿舍狭窄而拥挤,院子也不过是一片延伸到城墙边的永远被阴影笼罩的草坪。宿舍内吵闹不堪,旁边的工厂日夜两班替换交接,只有每晚从午夜至凌晨4点这4个小时停工。每周工作6天,空气潮湿发霉,300个工人和20多匹马散发出的难以忍受的臭味,令人无法久留。...

...

于是造币厂成为旧币再生的机房,每件事都讲求效率——吃进旧币、吐出新币的效率。古老的方法是用人力锤击来制造硬币,这是一种劳动力密集的生产方式,工作进度缓慢且效率低;而新式的压制机原理是由法国的布隆多引进的,自查理二世时代开始即以小规模操作。在由50匹马供应动力的10个机房中,巨大的滚筒轧出厚度精确的金属薄片,自薄片剪下空白硬币送进压制机中,那台压制机是由几个工人推动的一根大转轴,轴两端各有一把重锤,一个无助的少年硬币工将空白硬币一次一个地送进压制机下的机槽内,然后那把重锤随即击下,把皇族的肖像压印在硬币面上。可怜的少年工做不了几天就会损失至少一根手指头。最后,再由铣工将硬币的边缘铣出不易剪开的花纹。

牛顿最关心的事是提高工作效率。他仔细观察制作过程中的每一个步骤,制定工作系统的时间动作分析表,找出在何处以何种方法可以加以改善。他发现如果压制机的重锤运动与少年硬币工的动作相互配合适当,一个硬币工可以每分钟做50~55次入料和取出。他有几本记载造币厂工艺的记事簿,里面详细地记录分析了制造钱币的过程。

据在造币厂工作的职员海恩斯所写的钱币铸造经过,他认为牛顿所下的功夫是使那项作业成功的根本原因。牛顿的数学技巧使生产的过程流畅,这大大地提升了工作效率,并且“他能评定工人的勤惰”,这点可以想象得到。

...

牛顿之前的科学大师是不会这么做的。如果说牛顿对下属真的严格,那么他对自己比对下属要求更苛刻,他并不满足于把这个闲差职位变成全职的工作。就在到达弥漫着狂狷之气的造币厂和财政部之后,短短几个星期内,牛顿已成为这些机构中举足轻重的人物。尽管在生产高峰时期每天工作16个小时,牛顿仍然有时间和精力去取回本来属于他的职权——被他视为当仁不让的权力。

造币厂的总监尼尔并不是真正的阻碍,他极满意于他那终身任职的舒适工作,根本不在意牛顿逐渐加强自己的权责。在牛顿看来,多位前任厂长对工作的疏怠导致这个职位一度拥有的权力在逐渐失去。牛顿一生总是喜好权力与地位,他会去争夺或谋取,如今他既然握有了那一点儿权力,就会尽力再往上构筑。造币厂厂长的头衔只是踏脚石,是他攀登社会阶梯计划中的第一步。

...

为了达到这个目的,牛顿同时开始向两个方向行动。他先把旧的法令规章和政府文案找出来,以研究造币厂内部的职权划分。然后,他扩大了对职权的解释,只要曾经属于厂长职责的,他都不会放过。这里隐约可看见阿里乌的阴影作祟,他向“三位一体”的说法挑战时,即是先阅览一遍《启示录》,从《但以理书》中参悟出预言。此刻,他再度为需求寻找支持的证据,但这次不是为了圣言之战,而是为了权力之争。造币厂的文件中没有上帝,但确确实实有提高牛顿自我的东西。

他一找到事实真相之后,就立即将其呈送给财政部里受到蒙蔽的长官,向他们详细地解释造币厂的功能和组织,指出厂长应有的权力几十年来因无知而被剥夺。他甚至还研读了能拿到手的每一本经济学论著,吸取最先进的财政思想家的见解,包括布鲁斯特、朗兹、布瓦扎尔,以及他的朋友洛克。他特别珍惜由法国政府出版、装订成一套的《货币制作》,里面收录了180份官方文件。当后人整理牛顿的图书室、进行编目的时候,依工作人员所描述的书本状况,我们觉得它像是牛顿最常翻阅的书籍之一(在他去世后,人们归纳出他最常用的书籍中有31册属于经济方面的论著)。

就如早年追寻点金石和追究“三位一体”论为谎言的时候一样,牛顿一旦收集到足够有用的资料就开始写作,用他有限的空闲时间写满一页页的经济史、商业理论、各国的钱币制度及原理等。他将造币厂里的组织系统和已被混淆的权责关系用图表来说明,并仿效他的哲学记事簿和实验室记录的做法,在每一页上都列出了不同门类,例如在“关于造币厂的观察”的大标题下,又有“成分分析”“熔融技术”“制造钱币”等副标题。他雇用了几个抄写员整理他的草稿,并将他写好的每一篇文章都做了副本。据康杜伊特说,18世纪20年代,有一箱箱的文件被烧毁,其中有许多箱是牛顿与财政部的上司进行复杂争论时,所写的各种文件的副本。

可想而知,财政部当局一定被他的大量宣传品弄糊涂了,但当局并没有完全否认他所争取的权力。他们对尼尔的信任基本上可以确定已被新任厂长表现的高效率动摇了,因为很明显地,造币厂最近的成就和总监无关。牛顿也很可能得到了蒙塔古的勉励,蒙塔古则尽力为牛顿的许多请求铺平道路。
\end{bookref}



\section{打击伪币}
在\cite{牛顿与伪币制造者}中关于牛顿如何打击伪币和那段鲜为人知的侦探岁月有着详细的描述,这里就略过了。


\section{提出金本位制度}
牛顿爵士主要是因为出色地解决了前两个问题的工作而被封的爵士,但新币铸造计划却是一个注定失败的计划。后面牛顿就发现了不断地铸造新币,好像市面上仍然是旧币占绝大多数,而铸币厂后面却常常是连银子都找不到了。这在经济学上常称之为劣币驱逐良币现象。市场的力量是不容违背的,任何对抗市场规律的尝试,不管是谁,都将遭受惨败,这在历史上一次又一次的重演。

1717年,牛顿先英国议会提交了一份报告,分析了各国金银价格的情况,并提议将每盎司黄金的价格固定在3英镑17先令10.5便士,来解决前面提到的白银外流问题。这个提议被英国议会通过了,直到1816年英国议会通过《金本位制法案》正式在法律上宣告大英帝国的金本位制度。而在这段过渡时期,银子外流问题仍然存在,市面上流通的银币越来越少,所以牛顿的那段造币厂岁月让牛顿功成名就,但实际上从某种意义上来说这是一段失败的岁月。不过幸好晚年的牛顿没有骄傲自大,接受了这个失败,分析失败的原因,从而为金本位制的发展打下了地基。

货币是现代经济运行的基石,货币基础不打牢,之后的经济稳定运行和良好发展怕也是空中楼阁。此时的英国和其他国家,经济运行上一个基本情况就是货币供应严重不足制约了经济发展,确立金本位的反面意思就是放弃银本位,放弃历史传统的银币的银价值属性,这样为大众熟知的货币从价值含量载体更多地变成了一个价值符号,这就进一步让大众接受纸币打下了基础。只有纸币大行其道,才真正解决了现代经济的货币供应不足问题。并且随着纸币为大众接受,其他价值符号如债券股票等等才能被大众接受,如此现代金融才能继续发展下去。



\chapter{法国的债务危机}
\section{约翰·劳鼓吹国家发行纸币无果}
1705年约翰·劳出版了《论货币和贸易:兼向国家供应货币的建议》小册子,其中鼓吹了国家要繁荣就要发行纸币的观点。

引用自\cite{逃不开的经济周期}:
\begin{mdframed}
当时苏格兰的经济正处于不景气时期,劳相信自己看到了问题的症结所在:经济不景气与货币有关。这本册子还提出了一种从未有过的说法——“货币需求”。劳试图向读者说明,由于货币供给量太少,所以货币的利率就太高。解决的办法就是增加货币供给量。他声称,扩大货币供给量能够降低利率,而且,只要国家以全部生产能力运行,就不会导致通货膨胀。

他还提出了另外一项建议:在苏格兰建立一家“土地银行”。该银行可以发行银行券,但发行银行券的价值绝不超过国家所拥有土地的价值。持有银行券的人可以获得利息,并且有权选择在特定时间将银行券兑换成土地。这个新的方案有两方面的优点:

\begin{itemize}
\item 它将减轻国家的负担,即避免为了适应经济增长而购买越来越多的贵金属来铸造钱币。
\item 它将使国家更容易管理流通中的货币量,以便适应国家需求的变化。
\end{itemize}


这个建议非常好,产生了很大的轰动效应,同时也引发了争议。批评者嘲弄这是一个“沙滩银行”,将会破坏国家的命脉。但是,另外有一些人则支持劳的想法,最后,议会也对这个问题进行了严肃的辩论。然而,事情也就到此为止,大部分议员最终还是拒绝了这个方案。约翰·劳对此感到非常失望,再加上他又得不到英格兰法庭对他过失杀人罪的赦免(当时的英格兰与苏格兰是两个不同的国家),于是,他又回到欧洲大陆。...

但在劳的心里,还一直挂念着有关纸币的想法。他确信欧洲的繁荣需要纸币信用。大约1708年,他在法国的法庭上向一位检察官提出了建立土地银行的计划。然而,这个建议再一次被拒绝。之后,他又在意大利作过尝试,结果同样被拒之门外。
\end{mdframed}

\section{法国政府正陷入严重的债务危机}
1715年法国的路易十四去世,此时路易十五年仅7岁,真正执政的是摄政王奥尔良大公。由于路易十四对珠宝与宫殿的兴趣,挥霍无度,法国的财政已经是摇摇欲坠。

此时法国国家债务20亿里弗尔,年财政收入1.45亿里弗尔,而此时的法国每年财政支出1.42亿里弗尔,这些债务每年需要支付利息9000万里弗尔,相当于贷款利率4.5\%。

为了解决债务危机,奥尔良大公决定让硬币缩水,他下令所有硬币都要回收到造币厂,并禁止使用旧硬币,然后替换为新的硬币,新币含金量只有旧币的80\%。这种做法不得民心,对国家财政也仅仅贡献了7000万里弗尔。

奥尔良大公又发布政令,如果有人举报腐败的政府官员,其定罪后,举报人可以获得罚没财产的20\%。这个政策是令人高兴的,政府查抄了1.8亿里弗尔,将1亿里弗尔拨给了新官员,这样还剩8000万里弗尔。

上面这两个办法政府获得的总收入为1.5亿里弗尔,也仅仅填补了国家债务的7.5\%,此时的奥尔良大公绞尽脑汁,也无计可施了。

\section{劳氏公司}
1716年,奥尔良大公接见了约翰·劳,和他讨论了施政的政策。约翰·劳再次重复了以前说过无数次的话:要繁荣,就需要纸币,而且这种纸币还应该是硬通货,不贬值,不缩水。他提议设立一家银行来管理王室的收入,这家银行所发行的银行券要完全由贵金属或者土地作为支撑,换句话说,这是改良的“土地银行”。结果,大公高兴地同意了。

\begin{mdframed}
1716年5月5日,一家名为劳氏公司(Law \& Company)的银行创立了。银行从做担保业务开始,宣布所有税收都要用劳氏银行所发行的银行券缴纳,法国由此采用了纸币。

劳氏公司的资本为600万里弗尔,如果要购买其股份,需要用硬币支付其中25\%,其余75\%用行政债券支付。这是非常聪明的一步棋。行政债券是路易十四为给其巨额花销进行融资而发行的债券,这些债券在刚发行的时候售价为100里弗尔,现在则只值21.5里弗尔,因此被认为是垃圾债券。

债券的市场价值如此之低的原因当然是人们担心国家破产。然而,有一个可能的解决办法就是,政府按照当前很低的市场价格回购债券。如果这样做,政府实际上就可以将债务从20亿里弗尔(按政府出售债券时的价格算)减少到4.3亿里弗尔(按当前的市价算)。这样做实际上没有伤害到任何人!而且,此举有助于恢复信心。从原则上讲,政府还可以通过发行新的债券来支付债务利息——现在是按照4.5\%的利率来计算4.3亿里弗尔债务的利息,可见利息负担已大为减轻,新的利息负担大约仅为每年1900万里弗尔。

现在的问题是,政府如何设法把20亿里弗尔的垃圾债券收回,并且不至于抬高其市场价格。如果人们真认为菲利普·奥尔良会从价格挤压中摆脱出来,他们肯定就会提高债券的报价,这样一来,他的图谋就会失败。约翰·劳劝说人们专门用行政债券购买劳氏公司股票的办法,也只能解决一小部分问题。

在这个阶段,劳的“债务-股权互换”部分仅占债务总额很小的比例,剩余的政府债务还有18.5亿里弗尔。发售劳氏公司的股票所购回的行政债券仅为450万里弗尔,即600万里弗尔的75\%——相比20亿里弗尔的债务总额,几乎可以忽略不计。但是,劳已经有了下一步的计划,他又做了下面的三件事情:
规定劳氏公司银行券可以“见票即付”。这就是说,无论何时,只要你愿意来到劳氏公司,出示持有的劳氏公司银行券,都可以足额兑换硬币。

规定其银行券可以兑换旧币。如果政府采用缩水硬币(之前常常如此),那么约翰·劳仍然会支付原始含金量的硬币。

他公开宣称,任何银行家在发行银行券时如果没有足够的储备作为支持,就应该“受死”。

他这样做的结果是:新的纸币作为硬通货被接受,而且一开始的交易价格为101里弗尔,也就是说,与相同名义价值的硬币相比,还有1\%的溢价。这种可靠交易手段的出现,很快就刺激了贸易的发展:商业出现好转,对纸币的需求也与日俱增。不久,劳氏公司就在里昂、罗谢尔、图尔、亚眠和奥尔良开设了分支机构。

1717年,劳氏公司的纸币价格用硬币计价已经上升到了115里弗尔。

奥尔良大公开始对劳氏公司银行很感兴趣,他决定采用若干特许权——包括冶炼金银的唯一特权——进一步支持银行。他甚至还同意了一件从一开始就不愿意做的事情:将银行命名为“皇家银行”。很显然,这时他已经掌控了这家银行,而且高兴怎么干就可以怎么干。他之所以这样胆大妄为,是因为看到了以下四个方面的问题:

\begin{itemize}
\item 人们对纸币已经树立信心
\item 纸币是政府借款的“无痛方法”
\item 由于纸币处于溢价交易状态,很显然供给不足
\item 纸币似乎带来了繁荣兴旺
\end{itemize}

既然如此,为什么不多印发纸币呢?如果人们购买银行券来兑换硬币,大公就可以花费那些硬币!于是,他下令该银行印制10亿里弗尔的纸币——这超出了之前所印制纸币的16倍之多。这个命令遭到了大臣达古梭的反对,大公于是立即用更加听话的人取代了达古梭。约翰·劳对此感到了恐惧。
\end{mdframed}


\section{密西西比公司}
为了把剩余的行政债券消化吸收掉,约翰·劳进一步提出实施一项新的债务-股权互换计划。他建议奥尔良大公应该同意设立一家公司,这家公司获得与两个殖民地进行贸易的垄断权——这两个殖民地是1684年法国政府占领的密西西比河与路易斯安那州。在公司公开出售股权时,人们应该用行政债券购买,如此一来,国家的债务就消失了。大公对此提议非常兴奋,于是开始着手准备这个新的“密西西比计划”。

\begin{mdframed}
1719年年初,约翰·劳启动他的密西西比计划。新的密西西比公司的特许权得以扩大,包括:

\begin{itemize}
\item 在密西西比河、路易斯安那州、中国、东印度和南美享有贸易专权
\item 为期9年的皇家造币权
\item 为期9年的国家税负征收权
\item 烟草专卖权
\end{itemize}


除了这些,密西西比公司还获得了塞内加尔公司(The Senegal Company)、中国公司(The China Company)的全部财产,以及部分法国东印度公司(The French East India Company)的财产。随着法国东印度公司被控制,人们期望这个新的巨人能够挑战全能的英国东印度公司。

由于拥有了这些特权,不难想象,这家公司会创造出巨额的利润。公司被命名为“印度公司”后,又宣布增发价值2500万里弗尔的公众股票,从而使公司的总股本增加到了1.25亿里弗尔。约翰·劳对外宣称公司的预期红利能够达到5000万里弗尔,这就意味着投资年收益率达到40\%。然而,由于投资者是用“太阳王”的垃圾债券来购买股票的,所以实际上获得的投资收益率比40\%要高得多。

我们可以举一个例子来算一下,假如你购买价值100万里弗尔的股票:
\begin{Verbatim}
股票名义价格:100万里弗尔

预期年红利:40万里弗尔

用名义价值100万里弗尔的债券(折现率为0.2)购买的价格:20万里弗尔
\end{Verbatim}

因此,实际投资收益率竟然高达200\%!!!顷刻之间,申购股票的投资者蜂拥而至,股票被超额认购。公司职员要花几个星期的时间来整理认购人的名单。...越来越多的人挤满了坎康普瓦街,没过多久,人数就增加到好几千。这可不是一群普通的人,其中就有诸位公爵、伯爵和侯爵夫人,所有人都急切地想狠赚一笔。最终名单出来的时候,这次股票发行已经至少被超额认购6倍。而在自由市场上,股票价格急剧飙升到了每股5000里弗尔,是发行认购价格的10倍。劳和奥尔良大公决定好好利用人们的这种激情,于是又增发价值15亿里弗尔的股票,这次的发行规模达到了前两次的12倍之多。

\end{mdframed}

\section{密西西比泡沫}
\begin{mdframed}
这次股票发行确实应该引起投资者的担忧。我们这样想一想:投资者投入的是垃圾债券,并没有新的资本——仅仅是利息——注入公司,但是,随着资本份额的扩大,相应的每股收益已经被大大稀释,仅为原来的1/13。

然而,公众对此并不担忧,导致如此巨额的股票发行仍然有3倍的超额认购。于是,极为离奇的事情发生了。尽管4年前法国还陷在深深的绝望之中,但仅仅过了4年,整个国家又开始沸腾起来,到处充满了喜悦与幸福。所有的奢侈品价格都开始上涨,花边缎带、丝绸、宽布和天鹅绒的产量翻了几番。工匠的工资涨了4倍,失业率也下降了,到处都在忙着建造新的房屋。每个人都看到价格在不停地上涨,谁都想赶在价格进一步大幅上涨之前去抢购物品,抢着投资,抢着囤积。

在巴黎,经济比其他任何地方都要热。据估计,在此期间,巴黎的人口增加了30.5万。街道上常常塞满了新的马车,挤得谁也动弹不了。巴黎从世界各地进口了大量的工艺品、家具和装饰品,这种情形还从未有过,消费者也不再仅仅由贵族构成,还包括新兴的中产阶层。那些购买了股票的人突然发现,区区几千里弗尔居然可以增长到100多万里弗尔。很快,法语增加了一个新的词汇——“百万富翁”。不过,最大的受益者还是贵族阶层。其中当然包括约翰·劳,此外还有他的朋友理查德·坎蒂隆,坎蒂隆当时是法国巴黎最成功的一位银行家。

约翰·劳、坎蒂隆以及他的兄弟伯纳德一起在密西西比购买了16平方里格的土地,并且招募了大约100位想淘金的移民到那里种植烟草。不久之后,伯纳德就和他的移民一起乘坐贩奴船起程了。当他到达那里的时候,才发现实际情况并不像先前在巴黎的沙龙里所描绘的那样,而是布满了荆棘与敌意——在接下来的4年里,他带来的人有3/4死于疾病或印第安人的袭击。

然而,这类故事需要经过一些时日才能传回故乡,所以巴黎的投机狂热并没有丝毫减轻的迹象。先前在萧条中受到残酷压榨的许多中产阶层人士,如今依靠对印度公司的股票投机得救了。波旁公爵就是其中之一,他在股票交易上赚了大笔的钱,这足够让他在尚蒂伊重建一座无比奢华的宅第。他的投机还让他能够从英格兰进口150匹精心挑选的赛马以及购买一大片土地。其他许多中产阶层人士也都发了大财,但最大的玩家之一还是理查德·坎蒂隆,他是劳的朋友与投资合伙人,在股票价格还很高的时候,手里便累积了大量的股票。

...

随着牛市的继续,在劳位于坎康普瓦街的房子外发生了一些稀奇的事情:整条街都变成了股票交易场所,挤满了针对印度公司股价变化进行投机买卖的投机商。股票经纪人与中间商在这条街上到处租房子,租金比通常的价格高出12~16倍,而且连酒吧与餐馆也改成了股票交易场所。随着投机商和金钱而来的,还有小偷与骗子。所以,派一群士兵到坎康普瓦街来维持夜里的治安,已经是见怪不怪的事情了。

最后,劳实在受够了外面喧嚣的噪音与拥挤,于是在宽敞的凡登广场旁找了一个新的住处。但是,他不能从这些人当中搬走,因为在这些人的眼中,他是所有活动的中心。对他们来说,他比历史上任何国王都要伟大,是最伟大的金融天才,他独自创造了一个国家的繁荣新景象。贵族们用大笔金钱贿赂劳的仆人,就是为了能成为劳的听众。无论何时他驾车外出,皇家骑兵都要在前开道,为他挡开那些崇拜者。那些投机商与股票经纪人必须清楚地了解他的一举一动,就像圣·西蒙在他的回忆录中所写的那样:

\begin{quote}
劳被那些信徒与野心家紧紧围绕着,有的人把他的屋门挤坏了,有的人从他的花园翻窗而入,还有一些人从他办公室的烟囱上爬了下来。
\end{quote}

...

因此,就像工蜂追随着蜂王一样,人们也紧紧跟随着约翰·劳。不久,他家门前的广场上又搭满了摊位与帐篷,凡登广场也变成了一个兴旺繁忙的集市,人们不仅在这里从事股票与债券的买卖,还做起了各种各样的生意。广场上一片喧嚣,这比先前在坎康普瓦街时还要糟糕。奥尔良大公听到了一些对这种乱七八糟的状况的抱怨,尤其是首席法官的抱怨。因为他主持的法庭正好也在凡登广场,外面噪音已经让他听不清律师的讲话。约翰·劳决定再找个新的地方,于是买下了苏瓦松酒店,这个酒店的后面有一个大花园。就在同时,法庭发出了明文规定,除了这个花园之外,禁止在其他任何地方进行股票交易。于是人群再一次蜂拥着跟了过来,酒店的后面立即搭起了500多个大大小小的帐篷。这一次,巴黎的每一个人,不论男女老幼,几乎都在投机买卖印度公司的股票,而这只股票正处于加速的牛市行情。故事还在继续,当时清醒的阿贝·特诺松与和他同样清醒睿智的朋友拉莫特相互庆贺彼此都没有卷入这场全民的疯狂。然而,几天过后,拉莫特禁不住诱惑,跑去买了一些印度公司的股票。但是,当他走进苏瓦松酒店的时候,迎面碰见从酒店里面出来的人是谁呢?当然是阿贝,他刚刚在市场上买进了股票。在这个插曲之后的很长时间里,他们在经常进行的哲学讨论中都避免谈及投机的话题。

与此同时,奥尔良大公还在通过皇家银行印发更多的纸币。为什么不呢?难道不是发行货币的做法使国家重新繁荣起来的吗?既然是这样,为什么不多印发一些货币呢?打个简单的比方,货币对于经济这部机器来说就像油一样,不是吗?油灌得越多,机器就会运转得越好!这对股票市场也同样有好处。印度公司的股票价格已经从初始的每股150里弗尔飙升到超过8000里弗尔。就在这一天,一位生病的投机商听到如此令人难以置信的价格后,就打发他的佣人去卖掉250股的股票。当佣人来到市场,他看到价格实际上更高,卖出的价格不低于每股10000里弗尔,这已是原始发行价的67倍了——股票价格已经令人惊异地飙升了6700\%。他回来的时候,交给了主人400万里弗尔的预期收入。然后,他回到自己的屋子,收拾好东西,卷起剩下的50万里弗尔,迅速离开了这个国家。

1720年年初的一天,非常奇怪的事情发生了。一个人拉着两马车的纸币来到皇家银行门前,他愤怒了,而且非常愤怒……孔王子相信自己有充分的理由愤怒。他想购买一些印度公司的新股票,但是劳没有同意。这个傲慢的苏格兰杂种!踢开他!王子愤愤地骂道。于是,他拉着满满两车的纸币来到银行门前,径直走进了大门。“瞧,先生们!你们的纸币,所谓‘见票即付’的纸币。现在,你们瞧见了吗?那好,给我换成硬币吧!”银行随即把纸币换成了硬币,装了两马车。奥尔良大公听到这件事之后,显然大为震怒,立刻命令孔把2/3的金属硬币退回了银行。事情仅此而已。后来公众便不喜欢孔了,而且谴责他不合情理的做法。但是,这个事件仍然产生了重要的影响:它在民众的心里播下了一点点怀疑的种子。如果有更多人都拿着纸币要兑换,那会是什么样子呢?如果所有人都拿着纸币去银行兑换呢?银行会有那么多的黄金吗?我自己是否要去兑换呢?!

在随后的几个月里,一些机敏的投机商开始从股票市场抽身,卷走收益,而一些股票价格在短暂地摸高到每股10000里弗尔的水平后便开始下滑。有一对兄弟俩,鲍登与拉·理查蒂埃尔,开始悄悄地拿着纸币到皇家银行去兑换,每一次兑换的数额都比较少。他们还开始尽量收购白银与珠宝,并且把白银、珠宝和硬币一起秘密地运到荷兰与英格兰。一位成功的股票交易商沃默雷特也完全卖空了股票,把价值100万里弗尔的金属硬币装进了马车。他在上面覆盖了干草与牛粪,自己假装成农夫,驾着马车跑到了比利时。许多人离开了法国,剩下的人对纸币也越来越没有信心,并秘密储藏金属硬币。人们要么把硬币藏在床垫下,要么就把它们运到国外,这样一来,法国的货币流通速度慢了下来。

在这种情况下,大公采取的措施实在不够高明。首先,他把纸币兑硬币的兑换价调高了5\%。显然,他第一步是想恢复信心,但是这对资本外逃毫无效果,于是他把兑换价又调高了5\%,但还是不见效果。

1720年2月,他干脆禁止使用硬币。在法国,任何人财产中的硬币价值不得超过500里弗尔,否则就有被罚没充公的危险。他还禁止收购白银、珍贵宝石和其他珠宝。任何举报收购这类贵重物品的人,将会得到罚没财产价值的一半,当然这一半的价值是要用纸币支付的。最后,大公在2月1日到5月底的这段时间里,又印发了价值15亿里弗尔的纸币,纸币的总供应量已经达到了26亿里弗尔。很显然,公爵采取所有这些措施的目的,就是要迫使人们继续使用纸币,然而,这些举措此时已经回天乏力,毫无效果。经济已经开始紧缩,人们心里充满了恐慌。法兰西的未来在哪里呢?这又该谴责谁呢?

约翰·劳。正是这个约翰·劳应该受到谴责。不是他最先编造了纸币的故事吗?他的密西西比计划又怎样了呢?人们在那边除了被蚊子咬死或是被印第安人杀死,还能干些什么呢?印度公司的股票真的比皇家银行的纸币还值钱吗?难道不是这样吗?最好还是把这些东西统统卖掉吧!于是,股票价格很快崩溃了,大约超过50万人亏了本,成千上万的投资者破产了。那些在股票投机上亏本的人还不了别人的钱。面对这样一条残酷无情的反应链,需要采取一些补救的措施,以使印度公司的股东们相信公司实际上仍然运转良好。补救的办法很简单:把巴黎最穷的人与罪犯征召起来送到新奥尔良为公司挖黄金。有6000多个“穷鬼”参加了这个计划,这支队伍推推攘攘地在巴黎的街道上游行,准备去码头,然后坐船到美洲去。起初,人们喜欢这个计划,因为6000名工人已经是一支很大的队伍了。如果他们能找到金矿,那么公司当然会顺利运转。如果能用这些黄金铸造新的硬币,那甚至可以让法兰西再次振作和繁荣起来。于是,有一段短暂的时间,印度公司在股票市场上又重整旗鼓了。

但是,古怪的事情发生了:那些在街上游行的人绝大部分根本没有离开这个国家。3个人中就有两个把配发的新衣服与工具卖掉了,根本没上船,而是回到了家里。在巴黎忍受贫穷也比到新奥尔良挖黄金强。很明显,这个密西西比冒险计划已经不可能实现人们曾经寄托的希望。劳和他的朋友理查德·坎蒂隆也就放弃了从他们合伙购买的那块土地上挣钱的希望。

坎蒂隆对此显得很从容,因为他正在做着另外一桩挣钱的大买卖。当银行大量发行货币的时候,他的反应并不像其他人,实际上他早就看到了法国货币迟早是要贬值的,所以他收回了所有钉住法国货币的贷款,而把收益投在了英国货币上。劳听说了这种情况,于是来到坎蒂隆的办公室,威胁他如果不答应在48小时之内离开这个国家,今晚你就会被送进巴士底狱。

坎蒂隆马上卖光了全部资产,大约净赚了2000万里弗尔,这确实是一笔巨大的财富。然后他火速离开了法国。

这时,印度公司的股价还在持续下跌,奥尔良大公变得绝望起来。很显然,他越是采取限制使用硬币的措施,人们越是想要持有硬币。他决定把皇家银行与印度公司合并起来,希望两者能够相互支撑。可是这也没有奏效。

1720年5月初,他召集了一个约翰·劳和所有大臣都参加的紧急委员会会议。在会议的日程中,首要的便是处理正在流通的价值26亿里弗尔的纸币,而这每一张纸币都可以从官方兑换金币和银币。实际的硬币数额还不到一半,而且多数已经被民众藏在了床垫下面(法国人的这个习惯在此后几个世纪里已是臭名昭著)。会议决定将纸币贬值一半,从5月21日起生效。这对法国民众的打击简直太沉重了。由于社会动荡的不断升级和反抗的威胁,仅仅过了一个星期,即5月27日,原来的法令就被取消了。也正是在这一天,皇家银行暂停支付金属硬币,而约翰·劳也被解除了职务。

然而,这天晚上,大公派人去请劳,劳从一个密道进了王宫。大公竭尽所能地安抚劳,说劳这次成为众矢之的,被民众憎恨,是如何不公平。过了两天,他邀请劳去歌剧院看演出,劳还带着家人一起来,好让每个人都看到他们一家和大公在一起。但是,这对劳来讲,几乎是一个致命的错误。他的马车刚到家门口,民众就用石头进行袭击。车夫驾着车迅速躲进了大门,佣人随即把门砰地关上,劳才免遭皮肉之苦。劳受了惊吓之后,大公派了一队瑞士卫兵日夜驻扎在劳的宅子里。即使这样,劳还是感觉不安全。很快,他搬进了王宫,和大公享受同样的保护。

大公现在完全隐退了。为了帮忙收拾混乱局面,他决定重新起用两年前被他解职的大臣达古梭。为了能够劝他回来救场,他派劳坐着邮政马车去面见达古梭。达古梭同意了,而且和劳一起回来了。很快,在6月1日,禁止自由持有硬币的法令被废除。也是在同一时间,价值2500万里弗尔的新纸币得以发行,这些纸币是用巴黎的税收作为支持的。6月10日,皇家银行重新开张,也作好了纸币兑换金属硬币的准备——但它们已经不全是以往的贵金属硬币,现在有一部分被换成了铜币!

历史上,铜的价格经历了好多次牛市,但这一次更是独特。在接下来的几个月里,总有一群人聚集在银行门前,每个人都要把纸币兑换成一堆铜币。有几次聚集的人太多,以至于有人被挤死。为了缓解压力,7月9日,士兵封锁了大门,于是外面的人就开始投掷石块。一个士兵开枪还击,打死了1人,还伤了1人。8天之后,又有15个人因挤压而毙命。人们被激怒了,他们用担架抬着3具尸体游行到了皇宫花园。在这里,他们发现了约翰·劳的马车,于是就把它砸得粉碎。

委员会不得不寻求新的解决之道。下一个紧急措施就是进一步扶持印度公司,公司贸易特权的范围将进一步扩大,以至于垄断法国所有的海上贸易。这样做将使数千名独立的商人丢掉生意,于是议会收到了一封接一封满是怨言的请愿书。议会因此否决了这个方案。大公对此恼羞成怒,就把议会和所有议员驱逐到偏僻的蓬图瓦兹。

8月15日,一道新的法令强加到了可怜的法国人身上。该法令规定,除了购买年金、存入银行账户或者购买分期付款的印度公司股票之外,不允许进行全部纸币价值合计1000~10000里弗尔的交易。10月,印度公司的许多特权被拿掉了,纸币也贬值了。股东们被迫与公司一道持有股票,而且,那些已经同意购买公司新股票的人还被强迫按照几乎是当时市场估价30倍的价格购买。许多人试图离开这个国家,以逃避这恐怖的惩罚。于是,所有的边防哨所都接到了命令,要求扣留任何想出境的人,直到弄清楚他们是否认购了印度公司的股票。那些已经设法出境的人则要因缺席被判处死刑。

1720年法国有效货币供给下降有三个主要原因:

\begin{itemize}
\item 资本外逃。人们携带金币和银币离开法国。
\item 货币流通(速度)下降。人们因不相信纸币而储藏硬币,随后可能由于对每个人持有硬币数额的限制,人们更是竭尽所能地保存硬币。
\item 银行信用降低。法令强制规定,价值合计1000~10000里弗尔的所有纸币,只能用来购买债券、印度公司股票和存入银行账户,这就减少了有效货币供给。
\end{itemize}

约翰·劳现在整天生活在恐惧之中,他成了法国最遭憎恨的人。离开了皇家庇护所,他要么隐姓埋名,要么得找到一个强大的保护队伍。他请求搬到一个乡下庄园去,大公对此求之不得。几天后,他收到了大公的回信,大公在信中展现了仁慈,并且还允许他离开法国——如果他想离开的话。大公还同意送给他一笔钱,想要多少都可以,他恭敬地婉谢了大公的好意。随后,就在开启这场冒险旅程5年之后,他只带了一颗大钻石,离开了法国前往威尼斯,这一年他49岁。
\end{mdframed}

\section{南海泡沫}
南海泡沫也就比密西西比泡沫稍晚几个月爆发,将它们放在一起将最基本的逻辑就是密西西比泡沫爆发后资金从法国逃离到英国,进一步推动了南海泡沫。而专门研究金融史的书籍比如《千年金融史》将它们统称为1720年泡沫,认为这场泡沫的本质是跨大西洋贸易股份公司泡沫。除此之外南海泡沫和密西西比泡沫还有其他共性。

英国政府也被不断增加的巨额公共债务紧紧地缠住,其解决问题的措施也与法国类似。“南海公司”接管了偿付政府债务的义务,作为回报,它被授权垄断与南美的贸易。

前面提到的理查德·坎蒂隆,成功从密西西比泡沫破裂之前逃离到英国之后,又将资金投入到了南海公司中。1720年6月,南海公司的股价达到了历史顶峰,再一次,理查德·坎蒂隆又一次成功从顶部逃离。而在接下来的3个月里,股价下跌幅度达到了85\%,它也像法国的印度公司那样崩溃了。

许多投资南海公司的人是靠借钱来购买股票的,由于股票价格的崩溃,他们也失去了偿付债务的能力。于是造成银行倒闭的恐慌,结果拖累了很多金融机构的经营,导致了违约高潮的出现。

英国南海公司的结局又怎么样呢?它最终在1855年解散,其股票转换成了债券。在南海公司存续的140年时间里,它从来没有在南海做过什么辉煌的贸易。


\cite{逃不开的经济周期} 在讨论中有意将密西西比泡沫和约翰·劳的鼓吹纸币联系起来,这是不对的。南海泡沫之前的英国已经建立了一套运转良好的纸币制度了。1720年泡沫中的法国、英国、荷兰各有各的差异性,这些国别的差异性显然不是1720年泡沫的本质部分,我们应该关注和分析的是1720年泡沫中的那些跨越国别的经济中的共性问题。


\chapter{1763年巴黎和约}
1756-1763年,英国法国进行了七年战争。1763年英国法国签署《巴黎和约》,标志着关于世界海权的英法争霸以英国胜利结束。英国成为了当时世界上最大的殖民国家。

\begin{bookref}[frametitle={\cite{全球通史}}]
七年战争的海外方面的局势是由美洲的魁北克和印度的本地治里的陷落决定的。但是,欧洲的战争一直拖延到1763年即交战国缔结《巴黎和约》时。在美洲法属殖民地中,法国仅保有南美洲的圭亚那、纽芬兰沿海无足轻重的圣皮埃尔岛和密克隆岛以及包括瓜德罗普岛和马提尼克岛在内的西印度群岛的少数岛屿。因此,英国从法国得到了整个圣劳伦斯河流域和密西西比河以东的全部地区。西班牙于战争晚期站在法国一边参战,因此,被迫将佛罗里达割让给英国。作为补偿,法国把路易斯安那西部即密西西比河以西地区给予西班牙。在印度,法国人保有他们在本地治里和其他城市的商业设施——事务所、货栈和码头。但是,他们被禁止修筑防御工事或与印度王公缔结政治联盟。也就是说,法国人是作为商人而不是作为帝国建立者回到印度的。

《巴黎和约》签订时,英国政治领袖霍勒斯·沃波尔评论道:“烧掉你们的希腊和罗马书籍吧——那些有关微不足道的人们的历史记载。”这句颇有远见的评语有力地说明了这一和平解决办法的长远的、世界性的含义。就欧洲而论,条约允许普鲁士仍占有西里西亚、成为奥地利争夺德意志领导权的对手。然而,对世界历史而言,更重要的事实是法国丢失了北美洲和印度。这意味着格兰德河以北的美洲以后将发展成为英语世界的一部分。

法国被逐出印度也是一个具有世界意义的历史事件,因为它意味着英国人将在那里代替莫卧儿人。英国人一旦在德里安顿下来,就完全走上通往世界帝国和世界首位的道路。正是由于幅员辽阔、人口稠密的次大陆所提供的这块无与伦比的根据地,英国人才能在19世纪扩张到南亚其余地区,然后远远地扩张到东亚。出于这些原因,1763年的条约深刻地影响了世界历史的进程,并一直影响到今天。
\end{bookref}



\chapter{1783年美国独立}
1763年英国打败了法国赢得了海上殖民霸权,然而不到十几年,1775年北美独立战争爆发,1783年美国独立。


\chapter{1785年第一台蒸汽机驱动的纺织厂问世}
1785年第一台蒸汽机驱动的纺织厂在英国诺丁汉建成,标志着纺织业从水力时代进入蒸汽时代,这一标志性事件可以看作第一次工业革命的起点。


\chapter{时期结语}
从航海大发现开始,建立的那一套殖民商业经济模式,最终铸造了大英帝国的传奇和辉煌。大英帝国不是一夜衰落的,实际上当时的美国对于大英帝国经济版图来说并不是那么重要,其核心殖民商业经济模式受到的影响很小,不过衰落的种子在本时期最后几年已经能够看出端倪了:一个是原殖民地美国的独立;一个是通用动力蒸汽机的出现。

关于世界历史各个不同时期的划分,很多人有很多不同的见解,但以工业革命作为该时期的结束点我是很满意的,由此开启的工业经济发展时期和殖民商业经济时期很多逻辑都不同了,具体请看下面的讨论。



\part{工业经济发展时期}
\chapter{时期概览}
时间是1785年至今


\begin{enumerate}
\item 能够继续原来的对其他地方掠夺式的殖民商业经济模式,那么会继续实施。但已经放弃对其他地区的完全殖民掌控了,而更希望该地区继续落后下去,其工业发展不起来,不构成自身威胁。
\item 为了让其工业发展不起来,某些地区会被半殖民化,政府的经济关税等政策被彻底操控。
\item 利用金融工具对其他地区国家进行掠夺,如果某个地区被半殖民地化,其政府的银行金融等政策会被彻底操控。
\item 工业技术领先优势会慢慢扩散,为了保持自己的领先优势,会要求本国政府建立更好的大众教育体系来培养出人才。
\end{enumerate}

英国最先在工业领域建立领先优势,之后发展起来的国家要慢慢追赶上,无一例外要针对上面谈论的问题,要求本国政府做到:

\begin{enumerate}
\item 第一步,本国国家主权必须是独立的,政府官员是独立的,是考虑本国利益的,不是被外国操控的。
\item 为了保护本国工业,本国国家政府要建立独立的关税政策。
\item 为了保护本国金融,本国银行货币等金融政策必须是独立的。
\item 建立良好的大众教育体系,培养人才,利用这些人才来积极发展本国工业体系。
\end{enumerate}

有了上面的步骤,后起国家不一定能做到赶超原来发达国家,但至少可以让本国发展走上正规。

下面将会看到各个后起的国家,无一例外,都发生了或大或小的革命,那些传统的封建君主制政府,其官员都是为了自身利益而谋划,是根本做不到独立,只为了本国利益而行动的。做不到这第一步,后面的所有步骤都是不可能实现的泡影。



\chapter{第一次工业革命}





\chapter{第一次世界大战}


\chapter{第二次世界大战}

\chapter{美苏争霸}


\chapter{第二次工业革命}
第一次工业革命是节省劳力,第二次工业革命是节省脑力。





\appendix
\part{附录}
\chapter{词条解释}
\section{英国清教徒}
英国清教徒运动作为一次宗教改革运动,对于英国国教的批判,同当时英国国内政治,资产阶级革命活动是相辅相成的。清教徒中的长老派代表英国的大资产阶级;而清教徒中的独立派代表当时英国的中小资产阶级。比如独立派的领袖人物克伦威尔,上台对长老派进行了清洗,进行了共和革命。关于那段历史,总的来说都可以归结为当时英国的国内政治斗争史,更多的是政治诉求,而不是宗教信念上的东西。


\chapter{数据参考}



\backmatter
\chapter*{参考资料}
\addchtoc{参考资料}
\begin{thebibliography}{99}
\bibitem[全球通史]{全球通史} 《全球通史:从史前到21世纪:第7版新校本》by [美]斯塔夫里阿诺斯 at 2020.
\bibitem[全球经济史]{全球经济史} 《全球经济史》 by [英]罗伯特·C·艾伦 at 2015.
\bibitem[牛顿新传]{牛顿新传} 《牛顿新传》by [英]罗布·艾利夫 \& 万兆元[译] at 2013.
\bibitem[牛顿与伪币制造者]{牛顿与伪币制造者} 《牛顿与伪币制造者:科学巨匠鲜为人知的侦探生涯》by [美]托马斯·利文森 \& 周子平[译] at 2018. 
\bibitem[牛顿传]{牛顿传} 《牛顿传》by [英]迈克尔·怀特 \& 陈可岗[译] at 2020.
\bibitem[美国四百年]{美国四百年} 《美国四百年:冒险、创新与财富塑造的历史》 by [美]布·斯里尼瓦桑 \& 扈喜林[译] at 2022.
\bibitem[逃不开的经济周期]{逃不开的经济周期} 《逃不开的经济周期》 by 拉斯·特维德 \& 董裕平[译] at 2012.
\end{thebibliography}


\end{document}


