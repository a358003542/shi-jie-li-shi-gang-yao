% !Mode:: "TeX:UTF-8"

\documentclass[12pt,oneside]{book}

\usepackage{mybook} 
\usepackage{mybookcover}


\title{世界历史纲要}
\author{Wander}
\hypersetup{
    pdftitle={世界历史纲要},
    pdfauthor={Wander},
    pdfcreator={Wander},
    pdfsubject={历史},
}
  

\begin{document}
\makemytitle

\flypage{感谢上帝}


\frontmatter 
\addchtoc{序言}
\chapter*{序言}
历史作为人文学科,还真是一门作者主观性很强的领域,我无意于历史学研究,也无意于要发表出什么惊世骇俗的观点,本来就想随便找一本世界通史,对世界历史基本脉络有个了解,然后择个人感兴趣的重点内容,以为节点,挂在对应的位置上即可。

这对我投资认知世界还是很有必要的,个人也无意要说出一个观点,让大家都拍手叫好,对对对,是这样的,个人觉得对,就行了。但找来的历史书籍大多都不让我满意:

\begin{enumerate}
\item 我对太古早的那些历史琐事不感兴趣,但糟糕的是大多历史书籍都在讨论几百年前甚至几千年前的一些事情,真实性不论,就算是真的大多对现在也是影响微乎其微的了。
\item 一般的世界通史对最近的历史有所谈及,但讨论的内容太少了太浅了,根本不够用啊。
\item 别指望有一本客观的历史书籍,包括那种大纲性质的,历史作为人文学科,作者风格主观叙事倾向都太明显了,我想找一本够客观的,差不多够用就行,但想一想觉得还是算了吧,别幼稚了,这怎么可能。
\item 在叙事上我可能会更加关注经济上的一些事情,甚至会将其看作一个核心底层逻辑,几乎很难找到那种对我胃口的叙事风格的书籍。因此本书如果一定要分类,我会推荐分类到 \verb+经济 - 经济史+ 。
\end{enumerate}





\addchtoc{目录}
\setcounter{tocdepth}{2}    
\tableofcontents



\mainmatter

\part{前言}
\chapter{历史的原则}
下面讨论的原则是普遍性的一般性的原则,偶尔出现的某个极其罕见的特例不影响这里的讨论。这些特例从整个历史大视角来看也是几乎可以忽略不计的。


\section{原则一:供求关系}
\uwave{供求关系决定了东西的贵贱价值。}

即使一个人不愿意为这个东西支付任何货币或者其他价值物,暴力获得该东西,该东西在此人心中的贵贱价值也不会受到半点影响,甚至没有货币和市场,也不影响这里的讨论。东西的贵贱价值仍由这个东西的潜在市场供求关系决定。这就是本原则称之为原则的原因,坚如磐石。


\section{原则二:趋利避害}
\uwave{趋利避害作为人的天性将主导一切。}

即使是一个疯子,再不怎么理性的人,蠢蛋,其思维行动的底层逻辑仍然是趋利避害,至少他自己是这样以为的,这就是本原则称之为原则的原因。

人们常常将这种现象更学术地称之为一个人说话做事是具有阶级局限性,这种阶级局限性是无法打破的,因为其底层逻辑就是原则二:人性的趋利避害。


\section{原则三:从众心理}
\uwave{一个观点说的人越多越看起来更正确。}

比如一个餐馆,生意兴隆去的人很多,那么人们会自然地认为这个餐馆的菜很好吃。从众心理也是广告总是保持其有效性的底层逻辑。人们常常将这种现象更学术地称之为一个人说话做事是具有时代局限性的。

对于原则三,人们会寻找各种特例,现象来说明原则三并不是那么坚不可摧。然而非常遗憾,历史现象看的越多,就会越来越发现,原则三坚如磐石。很多时候人们只是觉得似乎原则三的某一小部分被打破了,但即使是看起来被打破的那一小部分,也仅仅只是一种表象。处在整个历史洪流中的某个时代的某个人,身上打上了大量的烙印,这些烙印皆是因为从众心理。

对于从众心理推动的历史现象,应该从高于单独个体人的角度来看,即从整个社会的角度出发来看待这些现象。这也正是下面要讲的:社会驱动原则。



\section{原则四:社会驱动}
\uwave{历史现象是由社会整体驱动的。}

一般在描述历史现象的时候,人们都会夸大其词某些历史英雄人们,而如果这样表述,社会觉得应该是这样的,社会觉得这样会更好,人们会觉得实在是无稽之谈。对于整个社会的运作,不管是基于某种先验的作用,还是后验的实在的组织运作机制,将整个社会作为一个整体研究对象,总是比从个体人的角度出发要好一些,更贴近真实一些。这就好比人体内的细胞之于整个人体,从细胞的角度去理解整个人体运作之奥妙,是不可及的。


\section{最后的原则:丛林法则}
\uwave{丛林法则或者说:没有规则。}

动物的世界的法则就是弱肉强食丛林法则,人作为一种行为上更自由更脱离基因本能约束的动物,其构成的社会,假定人会比动物更加遵守某种规则,是没有道理的。更合理的推断是人类构成的社会比起某些动物同类构成的社会更加容易滑向彻底没有规则的丛林法则。之所以社会呈现出你看到的模样,是因为国家社会存在着各种奖惩机制,在这些奖惩机制照顾不到的地方,只有丛林法则。


\chapter{历史的分析}
\section{分析一:最低生活开支}
历史的分析需要跨越国家地区,跨越历史各个时期,很多情况都是不同的,可以利用最低生活开支 $ a $ 来获得某个通用性的数据。

基于最低生活开支,或者说贫困线的数据分析,对于某个工人的工资收入,将表述为最低生活开支的某个倍数。

\section{分析二:历史惯性}
一般来说关于一个地区跨越不同时代的数据,常常会发现某些东西很难改变,那么称这个东西具有某种历史惯性。

一个地区社会内部的分配制度常常具有很强的历史惯性。


\section{分析三:平均收入}
比如一个地区的平均收入是最低生活开支的 $ n $ 倍,这个数值和该地区的经济整体发展情况,社会分配情况密切相关,一般会具有很强的惯性。

这个 $ n $ 值有几个特定含义的标志值:

\begin{enumerate}
\item $ n $ 略微大于1,该地区大部分人陷入完全贫困陷阱中,地区经济状况具有较强的历史惯性,会一直被约束在略微大于1的位置,难以改变。
\item $ 1< n < 2 $ ,2这个数字意味着,该地区大部分人的收入勉强维持最低的生育延续线,该地区经济状况具有较强的历史惯性,任何进步因素都会被扼杀在摇篮里。
\item $ 1< n < 3 $ ,这个情况和上面的情况没有本质上的差异,属于该地区的好时候了,人口在增长,如果该地区文化有男尊女卑的倾向,那么这个时候女性会更加倾向于居家。该地区经济状况具有较强的历史惯性,但可能有一些进步因素在萌生。中国地区有名的历史周期律,其表现就是人们的平均收入在王朝好的时候会达到3点几的样子,然后就调转趋势向下陷入贫困中。
\item $ 2 <n < 4$ ,这个情况和上面的情况一样整个经济都属于农业型经济,不同的是该地区内部分配制度更公平政治更清明,会让这个数值更高一点,更加稳定在3到4之间波动。
\end{enumerate}






\part{西欧崛起的时期}
时间是1500年到1763年。


\begin{bookref}[frametitle={\cite{全球经济史}}]
这一阶段始于哥伦布和达伽马的航海探险(正是这些航行最终促成了一体化的全球经济),终于工业革命。在此三百年间,美洲开始被殖民并向欧洲输出白银、糖和烟草;非洲人被运往美洲充当奴隶,生产上述产品;亚洲则将香料、纺织品和瓷器运往欧洲。主要欧洲国家不断获取新的殖民地,并采用关税和战争手段阻挠其他国家与殖民地开展贸易,企图以此来增加本国的贸易收入。以牺牲殖民地的发展为代价,欧洲制造业取得了进步,但当时经济发展本身并不是各国的兴趣所在。
\end{bookref}




\part{西方占据优势的时期}
时间是1763年到1914年

\begin{bookref}[frametitle={\cite{全球经济史}}]
在这一阶段,情况发生了变化。当拿破仑于1815年兵败滑铁卢时,英国已经在工业领域内确立了领先优势,将其他国家都甩在身后。西欧各国和美国将经济发展作为首要任务,试图采取一系列政策来发展经济。这些政策包括四个部分:(1)消除内部关税,加强交通基础设施建设,从而建立统一的全国性市场;(2)建立外部关税,保护本国工业,应对来自英国的竞争;(3)成立银行,稳定货币并为工业投资提供资金;(4)建立大众教育体系,提升工人的能力。这些政策在西欧和北美取得了成功,这些地区内的各个国家和英国一起构成了如今的富国俱乐部。一些拉美国家没有完全采取这些政策,未能取得巨大成功。来自英国的竞争使得大多数亚洲国家没能走上工业化道路。自从英国的奴隶贸易于1807年停止后,非洲改为对外输出棕榈油、可可粉和各种矿物。
\end{bookref}

\chapter{第一次工业革命}





\part{1914年至今}

\begin{bookref}[frametitle={\cite{全球经济史}}]
到了20世纪,曾经在西欧(尤其是德国)和美国取得成功的各项政策在不发达国家实行后,并没有取得显著效果。大多数技术都由富裕国家发明,由于这些国家的劳动力价格日渐昂贵,所以它们需要发明新的技术,通过投入更多资本来提高生产率。对于工资水平较低的国家来说,采用新技术常常不够划算,但要想追赶西方国家,它们需要这些技术。大多数国家一定程度上都采用了现代技术,但它们的发展速度还不足以赶超富裕国家。也有一些国家在20世纪成功缩小了与西方国家之间的差距,它们采取的是大推进式的发展模式,通过计划手段和投资协调来取得快速发展。
\end{bookref}

\chapter{第一次世界大战}


\chapter{第二次世界大战}

\chapter{美苏争霸}


\chapter{第二次工业革命}
第一次工业革命是节省劳力,第二次工业革命是节省脑力。





\appendix
\part{附录}
\chapter{术语}



\backmatter
\chapter*{参考资料}
\addchtoc{参考资料}
\begin{thebibliography}{99}
\bibitem[全球通史]{全球通史} 《全球通史:从史前到21世纪:第7版新校本》by [美]斯塔夫里阿诺斯 at 2020.
\bibitem[全球经济史]{全球经济史} 《全球经济史》 by [英]罗伯特·C·艾伦 at 2015.

\end{thebibliography}


\end{document}


